\section{Objectif du Stage}

Ce stage a pour objectif, le développement d'une bibliothèque traitant 
d'automates à partir d'expression rationnelle en \textit{Haskell}. De plus, 
cette bibliothèque doit être accompagnée d'une description formelle. Que ce soit 
des algorithmes et même des structures utilisées. Il a de plus été envisagé 
assez tôt dans ce stage, la mise en place d'une interface graphique qui pourrait 
améliorez l'accessibilité de cette bibliothèque.

\vphantom{}

La bibliothèque d'automate ne devait pas seulement permettre de créer un 
automate, elle devait aussi mettre en place des fonctions usuelles sur les 
automates. Toutes ces fonctions sont détaillées dans la spécification technique 
de la bibliothèque. De même, le développement d'un parser de chaine de caractère 
vers une expression régulière et d'un fichier \textit{JSON} vers un automate 
devait être mis en place.

\vphantom{}

Pour améliorer la lisibilité, une représentation graphique à l'aide du langage 
\textit{dot} et de ces utilitaires devait être mis en place. Enfin, un site 
web avait était prévue pour mettre en place une interface graphique améliorant 
l'accessibilité de toutes les fonctionnalités citées précédemment. 

\subsection{Organisation du Travail}

Le développement de cette bibliothèque se fait parallèlement en \textit{Haskell}
et en \textit{Rust}. En effet, nous sommes un binôme sur ce projet. Bien que les 
objectifs soient les mêmes, les implémentations n'ont rien à voir. La différence
entre les deux langages est si grande que très peu de ressemblance peuvent être 
trouvé.

\vphantom{}

L'organisation du travail pour ce stage, a été mis en place lors de notre 
première réunion qui s'est tenu le premier jour de ce stage. Il a alors été 
convenu de mettre en place une réunion hebdomadaire. De ce fait, toutes les 
semaines, une réunion pour parler des avancées, difficultés et objectif ont été 
mises en place. De manière générale, de telle réunion avait un temps variable 
pouvant aller de 20 minute à pour les plus longs plus d'une heure.

\vphantom{}

J'ai pour ma part après cette première réunion qui nous as donnés les objectifs
de ce stage, mis en place un diagramme de \textit{Gantt}, celui de la 
figure~\ref{fig:gantfig}. Comme on peut le voir sur cette figure, chaque couleur 
corresponde à une tâche en particulier. Ce diagramme aillant été fait au début 
de mon stage, certain objectif ont pu prendre plus de temps que prévue. La 
correspondance couleur, objectifs est celle ci-dessous~:

\vphantom{}

\begin{itemize}
  \item[\textbullet] \textcolor{red}{Rouge}~: Mise en place d'un type 
  d'\textbf{expression régulière}, ainsi que d'\textbf{automates}. Puis 
  implémenter la conversion d'expression vers automate à l'aide de l'algorithme 
  de \textit{Glushkov}.
  \item[\textbullet] \textcolor{blue}{Bleu}~: Développement d'un \textbf{Lexer}, 
  aillant pour but de convertir une chaine de caractère en \textbf{expression 
  régulière}.
  \item[\textbullet] \textcolor{green}{Vert}~: Développement de diverses 
  fonctions sur des automates. Que ce soit des \textit{`getters'/ `setters'} ou 
  encore des tests de propriétés comme l'homogénéisée ou encore savoir s'il 
  est standard (se référer au document technique).
  \item[\textbullet] \textcolor{yellow}{Jaune}~: Création d'une conversion d'un
  automate en \textit{Dot}. Pour une représentation graphique qui as pour d'être 
  d'être, elle aussi, développer.
  \item[\textbullet] \textcolor{purple}{Violet}~: Mise en place du site web qui 
  devait permettre d'obtenir une représentation graphique et une meilleure 
  accessibilité.
  \item[\textbullet] \textcolor{orange}{Orange}~: Enfin, ces deux dernières 
  semaines ont été laissées pour peaufiner chacune des parties précédentes, mais 
  aussi de rédiger ce rapport de stage.
\end{itemize}

\begin{figure}[H]
  \begin{ganttchart}[
    hgrid,
    vgrid,
    x unit=2.5mm,
    time slot format=isodate
    ]{2024-04-22}{2024-06-22}
    \gantttitlecalendar{month} \\
    \ganttbar[bar/.append style={fill=red}]{}{2024-04-22}{2024-04-26} 
    \ganttbar[bar/.append style={fill=red}]{}{2024-04-29}{2024-05-03} 
    \ganttbar[bar/.append style={fill=blue}]{}{2024-05-06}{2024-05-07} 
    \ganttbar[bar/.append style={fill=green}]{}{2024-05-13}{2024-05-17} 
    \ganttbar[bar/.append style={fill=green}]{}{2024-05-21}{2024-05-22} 
    \ganttbar[bar/.append style={fill=yellow}]{}{2024-05-23}{2024-05-24}
    \ganttbar[bar/.append style={fill=yellow}]{}{2024-05-27}{2024-05-31} 
    \ganttbar[bar/.append style={fill=purple}]{}{2024-06-03}{2024-06-07} 
    \ganttbar[bar/.append style={fill=orange}]{}{2024-06-10}{2024-06-14}
    \ganttbar[bar/.append style={fill=orange}]{}{2024-06-17}{2024-06-21} \\
    \ganttmilestone[milestone/.append style={shape=rectangle, fill=black}]{}{2024-04-22}
    \ganttmilestone[milestone/.append style={shape=rectangle, fill=black}]{}{2024-04-26}
    \ganttmilestone[milestone/.append style={shape=rectangle, fill=black}]{}{2024-05-02}
    \ganttmilestone[milestone/.append style={shape=rectangle, fill=black}]{}{2024-05-14}
    \ganttmilestone[milestone/.append style={shape=rectangle, fill=black}]{}{2024-05-23}
    \ganttmilestone[milestone/.append style={shape=rectangle, fill=black}]{}{2024-06-06}
    \ganttmilestone[milestone/.append style={shape=rectangle, fill=black}]{}{2024-06-13}
  \end{ganttchart}
  \caption{
    Diagramme de \textit{Gantt} mise en place pour le stage. Les rectangles 
    noirs correspondent au jour de réunion.
  }\label{fig:gantfig}
\end{figure}

% peut être encore parler de quelque chose mais je sais pas quoi

\subsection{Difficultés rencontrées}

J'ai pu au cours de ce stage, rencontrées de nombreuses difficultés. Pour les 
exposées, j'ai décidé de diviser en deux catégories. Les difficultés liées au 
développement, à la partie théorique et les problématiques spécifique d'un 
stage.

\subsubsection{Problème lié au stage}

Ce stage constitue ma première expérience proche d'un travail en entreprise. De 
ce fait, j'ai pu rencontrer les problèmes que je pense nombreux ont. Notamment 
pendent quelque jour, notamment la première semaine, nous étions en télétravail.
De nombreux problèmes survienne par ce mode de travail. Tous d'abord la façon de 
gérer ces horaires. M. Mignot nous as laissé gérer nos horaires nous même ce qui 
à fait en sorte que pendant le télétravail j'ai eu du mal à ne pas trop 
travailler et même dans l'autre sens parfois. Ensuite, on a les problèmes du 
quotidien comme des pertes de connexions qui rendent certaine journée 
pratiquement impossible à avancée. Enfin, je n'ai pas eu de chance lors de ces 
derniers mois et je suis tombé malade, ce qui a accentué encore plus ce 
télétravail.

\newpage

Un autre problème et survenue, celui-ci concernait le matériel. En effet, mon 
ordinateur portable n'avait selon mon système d'exploitation plus de mémoire 
vive. J'ai alors dù réinstaller un nouvel os. Cependant, je n'avais pas prévue 
qu'il existait encore des os qui ne se lancé pas sur \textit{systemd}. Sans 
rentrer dans les détailles, ce problème m’a pris énormément de temps à régler. 
C'est un problème, car le gestionnaire de paquet (s'il peut l'appeler comme ça)
\textit{Nix} (qui est l'outil le plus utilisé pour la création de site en 
\textit{Haskell} sous le paradigme de la \textit{Programmation fonctionnelle 
réactive}) doit ce lancé à l'aide de \textit{systemd}. Sait d'ailleurs 
une des raisons aillant poussé vers la migration du développement d'un site web 
vers une application \textit{Gtk}. Se référer à la sous-section suivante et à la 
partie résultat de ce compte rendu. 

\subsubsection{Difficultés de développement du projet}

Avant tout, j'ai n'ai commencé à apprendre ce langage que depuis décembre. 
Dès lors que j'ai obtenu ce stage et que j'ai appris que le sujet nécessite le 
\textit{Haskell}. J'ai donc demandé des ressources à Mr Mignot et j'ai suivi 
ces documentations. Notamment la lecture des livres 
\textit{Get Programming with Haskell} et \textit{Haskell in Depth}
\cite{bookWithHaskell, haskellInDepth}. Pour combler celà, j'ai aussi suivie 
certain cours vidéo disponible sur \textit{Youtube}
\cite{playHaksell1, playHaksell2}. De ce fait, malgré avoir commencé assez 
tôt l'apprentissage de ce langage, comme évoquer dans de nombreuse ressource la 
courbe d'apprentissage du \textit{Haskell} est exponentielle. Donc de nombreux 
problèmes que je vais évoquer ci-dessous aurait pu ne pas avoir vue le jour, si 
j'avais une plus grande expérience avec ce langage.

\vphantom{}

Le premier problème a été rencontré lors de la mise du convertisseur de chaine 
de caractère vers expression. Les modules utilisés ont été comme conseille par 
M. Mignot, \textit{Alex} et \textit{Happy}. Le problème a été la documentation 
de ces modules. Cette manière de documentation basée uniquement sur ce que font 
chaque fonction et non de comment on peut les utiliser et la non-présence 
d'exemple a rendu ce développement bien plus long. Ce même problème a été vu 
lors de l'utilisation des modules \textit{GraphViz} qui permet la représentation
d'un graphe. La documentation est totalement horrible. De plus, le module étant 
très peu utilisé il n'existe pratiquement aucun exemple sur internet. J'ai donc
du programmer a taton et me référer au code source du module pour comprendre 
son mode de fonctionnement.

\vphantom{}

Le deuxième problème, j'ai l'ai eu bien plus tard dans le développement. 
Cependant celui-ci m’a fait perdre pratiquement deux voir trois semaines. Pour 
la mise en place site web, il avait été convenu lors d'une réunion, d'utilisé 
le paradigme de la \textit{Programmation fonctionnelle réactive}. Ce paradigme, 
se base sur l'idée d'un tableur. Lorsque l'on modifie le contenu d'une case, le 
reste du tableau peut alors évoluer. C'est sur ce principe que repose la 
\textit{programmation réactive}. En \textit{Haskell}, il existe de nombreux 
modules qui permettent la mise en place de cette méthode de programmation. 
Cependant, celui qui est utilisé aux interfaces Web \textit{Reflex}, nécessite 
l'outil \textit{Obelisk}. Cet outil ma tout d'abord causée des problèmes quant 
à son installation, puis son utilisation. L'utilisation de \textit{Nix} 
par celui-ci, oblige à chaque installation, de compiler au moins une fois ce qui 
a été installer. De ce fait, son utilisation été très longue à mettre en \oe 
uvre. Après cela, il y a aussi l'utilisation du module \textit{Reflex}. Bien que
ce soit un module \textit{Haskell}, son utilisation requière un apprentissage
qui se rapproche a celui d'un langage. Notamment, car nous n'avons jamais vu ce 
paradigme et que donc ces mécanismes me sont inconnus. De ce fait, après avoir 
tout de même passer une semaine à essayé d'apprendre le \textit{frp}, je me suis 
redirigé vers la mise en place d'une application \textit{Gtk} (implémentation 
bien plus commune, qui permet donc un développement plus rapide).  

\vphantom{}

Enfin la dernière difficulté rencontrée à été la rédaction du rapport formelle 
dans un premier temps. Bien que j'aie entrepris son écriture en même temps que 
le développement de la bibliothèque. Le formalisme nécessaire et les définitions 
qu'il faut introduire pour que le document se suffise à lui-même a rendu cette 
tâche très chronophage. De plus, n'aillant jamais réellement été confronté a 
cet exercice lors de la licence, il a été dur de s'y adapté. Je me suis 
d'ailleurs rendu compte du nombre de relectures nécessaire pour un tel document. 
Je ne pensais pas qu'il fallait en faire au temps. De même pour la bibliographie 
qui doit si elle est citée définir les concepts de la même manière que ceux 
du document.

Cette difficulté de rédaction s'est aussi vue sur la fin de ce stage, lors de la 
rédaction de ce rapport. Aillant maintenant l'habitude des rapports de projet 
avec tous ceux rédiger lors de la licence, je ne pensais pas que celui-ci aller 
être si dure. La différence entre ce rapport et ceux précédent est surtout la 
différence de contenu. L'objectif de celui-ci ne vise pas à montrer toutes les 
facettes du projet sur laquelle j'ai travaillé, mais plutôt la manière dont j'ai 
travaillé déçu est ce que j'ai appris. 

