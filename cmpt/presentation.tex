\section{Objectif du Stage}

Ce stage a pour objectif le développement d'une bibliothèque traitant 
d'automates à partir d'expression rationnelle en \textit{Haskell}. De plus, 
cette bibliothèque doit être accompagnée d'une description formelle décrivant 
les algorithmes et structures utilisées. Il a de plus été envisagé 
assez tôt dans ce stage, la mise en place d'une interface graphique qui pourrait 
améliorer l'accessibilité de cette bibliothèque.

\vphantom{}

La bibliothèque d'automate ne devait pas seulement permettre de créer un 
automate, elle devait aussi mettre en place des fonctions usuelles sur les 
automates. Toutes ces fonctions sont détaillées dans la spécification technique 
de la bibliothèque. De même, le développement d'un analyseur (aussi appelé 
`parser' en anglais) de chaine de caractère vers une expression régulière et 
d'un fichier \textit{JSON} vers un automate devait être mis en place.

\vphantom{}

Pour améliorer la lisibilité, une représentation graphique à l'aide du langage 
\textit{dot} et de ces utilitaires devait être mis en place. Enfin, une 
interface graphique a été prévue, celle-ci améliorant l'accessibilité de toutes 
les fonctionnalités citées précédemment. 

\subsection{Organisation du Travail}

Le développement de cette bibliothèque se fait parallèlement en \textit{Haskell}
et en \textit{Rust}. En effet, nous sommes un binôme sur ce projet. Bien que les 
objectifs soient les mêmes, les implémentations n'ont rien à voir. La différence
entre les deux langages est si grande que très peu de ressemblance peuvent être 
trouvées.

\vphantom{}

L'organisation du travail pour ce stage a été mis en place lors de notre 
première réunion qui s'est tenu le premier jour de ce stage. Il a alors été 
convenu de mettre en place une réunion hebdomadaire. Cette réunion servait à 
évoquer les avancées et difficultés rencontrées lors de la semaine. De manière 
générale, de telles réunions avait un temps variable pouvant aller de 20 minute 
à pour les plus longues plus d'une heure.

\vphantom{}

J'ai pour ma part après cette première réunion qui nous a donné les objectifs
de ce stage, mis en place un diagramme de \textit{Gantt}, celui de la 
figure~\ref{fig:gantfig}. Comme on peut le voir sur cette figure, chaque couleur 
correspond à une tâche en particulier. Ce diagramme ayant été fait au début 
de mon stage, certains objectifs ont pu prendre plus de temps que prévu. La 
correspondance couleurs, objectifs est celle ci-dessous~:

\vphantom{}

\begin{itemize}
  \item[\textbullet] \textcolor{red}{Rouge}~: Mise en place d'un type 
  d'\textbf{expression régulière}, ainsi que d'\textbf{automates}. Puis 
  implémenter la conversion d'expression vers automate à l'aide de l'algorithme 
  de \textit{Glushkov}.
  \item[\textbullet] \textcolor{blue}{Bleu}~: Développement d'un 
  \textit{`parser'}, ayant pour but de convertir une chaine de caractère en 
  \textbf{expression régulière}.
  \item[\textbullet] \textcolor{green}{Vert}~: Développement de diverses 
  fonctions sur des automates. Que ce soit des \textit{`getters'/ `setters'} ou 
  encore des tests de propriétés comme l'homogénéité ou encore savoir s'il 
  est standard (se référer au document technique).
  \item[\textbullet] \textcolor{yellow}{Jaune}~: Création d'une conversion d'un
  automate en \textit{Dot}, qui permet une représentation graphique.
  \item[\textbullet] \textcolor{purple}{Violet}~: Mise en place d'une interface 
  graphique permettant d'obtenir une représentation sous forme de graphe de 
  l'automate obtenu.
  \item[\textbullet] \textcolor{orange}{Orange}~: Enfin, ces deux dernières 
  semaines ont été laissées pour peaufiner chacune des parties précédentes, mais 
  aussi de rédiger ce rapport de stage.
\end{itemize}

\begin{figure}[H]
  \begin{ganttchart}[
    hgrid,
    vgrid,
    x unit=2.5mm,
    time slot format=isodate
    ]{2024-04-22}{2024-06-22}
    \gantttitlecalendar{month} \\
    \ganttbar[bar/.append style={fill=red}]{}{2024-04-22}{2024-04-26} 
    \ganttbar[bar/.append style={fill=red}]{}{2024-04-29}{2024-05-03} 
    \ganttbar[bar/.append style={fill=blue}]{}{2024-05-06}{2024-05-07} 
    \ganttbar[bar/.append style={fill=green}]{}{2024-05-13}{2024-05-17} 
    \ganttbar[bar/.append style={fill=green}]{}{2024-05-21}{2024-05-22} 
    \ganttbar[bar/.append style={fill=yellow}]{}{2024-05-23}{2024-05-24}
    \ganttbar[bar/.append style={fill=yellow}]{}{2024-05-27}{2024-05-31} 
    \ganttbar[bar/.append style={fill=purple}]{}{2024-06-03}{2024-06-07} 
    \ganttbar[bar/.append style={fill=orange}]{}{2024-06-10}{2024-06-14}
    \ganttbar[bar/.append style={fill=orange}]{}{2024-06-17}{2024-06-21} \\
    \ganttmilestone[milestone/.append style={shape=rectangle, fill=black}]{}{2024-04-22}
    \ganttmilestone[milestone/.append style={shape=rectangle, fill=black}]{}{2024-04-26}
    \ganttmilestone[milestone/.append style={shape=rectangle, fill=black}]{}{2024-05-02}
    \ganttmilestone[milestone/.append style={shape=rectangle, fill=black}]{}{2024-05-14}
    \ganttmilestone[milestone/.append style={shape=rectangle, fill=black}]{}{2024-05-23}
    \ganttmilestone[milestone/.append style={shape=rectangle, fill=black}]{}{2024-06-06}
    \ganttmilestone[milestone/.append style={shape=rectangle, fill=black}]{}{2024-06-13}
    \ganttmilestone[milestone/.append style={shape=rectangle, fill=black}]{}{2024-06-18}
    \ganttmilestone[milestone/.append style={shape=rectangle, fill=black}]{}{2024-06-21}
  \end{ganttchart}
  \caption{
    Diagramme de \textit{Gantt} mis en place pour le stage. Les rectangles 
    noirs correspondent aux jours de réunion.
  }\label{fig:gantfig}
\end{figure}

% peut être encore parler de quelque chose mais je sais pas quoi

\subsection{Difficultés rencontrées}

J'ai pu au cours de ce stage, rencontrer de nombreuses difficultés. Pour les 
exposer, j'ai décidé de les diviser en deux catégories. Les difficultés liées au 
développement, à la partie théorique et les problématiques spécifique d'un 
stage.

\subsubsection{Problèmes liés au stage}

Ce stage constitue ma première expérience proche d'un travail en entreprise. De 
ce fait, j'ai pu rencontrer les problèmes communs. La première semaine, nous 
étions en travail à distance.
De nombreux problèmes surviennent par ce mode de travail. Tout d'abord la façon 
de gérer ces horaires. M. Mignot nous a laissé gérer nos horaires nous même ce 
qui a fait en sorte que pendant le travail à distance j'ai eu du mal à ne pas 
trop travailler et même dans l'autre sens parfois. Ensuite, on a les problèmes 
du quotidien comme des pertes de connexions qui rendent certaines journées
pratiquement impossible à avancée. Enfin, je n'ai pas eu de chance lors de ces 
derniers mois et je suis tombé malade, ce qui a accentué encore plus ce 
travail à distance.

\newpage

Un autre problème est survenu, celui-ci concernait le matériel. En effet, mon 
ordinateur portable n'avait selon mon système d'exploitation plus de mémoire 
vive. J'ai alors dû réinstaller un nouvel \textit{OS}. Cependant, je n'avais pas
prévu qu'il existait encore des systèmes ne se lancent pas sur \textit{systemd}. 
Sans rentrer dans les détails, ce problème m'a pris énormément 
de temps à régler. C'est un problème, car le gestionnaire de paquets (si on peut 
le nommer de la sorte) \textit{Nix} (un outil souvent utilisé pour la 
création de site \textit{Haskell}, sous le paradigme de la 
\textit{Programmation fonctionnelle réactive}) doit ce lancer à l'aide de 
\textit{systemd}. C'est d'ailleurs une des raisons ayant poussé vers la 
migration du développement d'un site web vers une application \textit{Gtk}. Se 
référer à la sous-section suivante et à la partie résultat de ce compte rendu. 

\subsubsection{Difficultés de développement du projet}

Avant tout, j'ai n'ai commencé à apprendre ce langage que depuis décembre. 
Dès lors que j'ai obtenu ce stage et que j'ai appris que le sujet nécessite le 
langage \textit{Haskell}. J'ai donc demandé des ressources à M. Mignot et j'ai 
suivi ces documentations. Notamment la lecture des livres 
\textit{Get Programming with Haskell} et \textit{Haskell in Depth}
\cite{bookWithHaskell, haskellInDepth}. Pour combler cela, j'ai aussi suivi 
certaines courtes vidéos disponibles sur \textit{Youtube}
\cite{playHaksell1, playHaksell2}. De ce fait, malgré avoir commencé assez 
tôt l'apprentissage de ce langage, comme évoqué dans de nombreuses ressources la 
courbe d'apprentissage du \textit{Haskell} est exponentielle. Donc de nombreux 
problèmes que je vais évoquer ci-dessous aurait pu ne pas avoir vue le jour, si 
j'avais une plus grande expérience avec ce langage.

\vphantom{}

Le premier problème a été rencontré lors de la mise en place du convertisseur de 
chaine de caractères vers expression. Les modules utilisés ont été comme 
conseillé par M. Mignot, \textit{Alex} et \textit{Happy}. Le problème a été la 
documentation de ces modules. Cette manière de documentation basée uniquement 
sur ce que fait chaque fonction et non sur comment on peut les utiliser et la 
non-présence d'exemples a rendu ce développement bien plus long. Ce même 
problème a été vu lors de l'utilisation des modules \textit{GraphViz} qui permet 
la représentation d'un graphe. La documentation est totalement horrible. De 
plus, le module étant très peu utilisé, il n'existe pratiquement aucun exemple 
sur internet. J'ai donc dû programmer à tâtons et me référer au code source du 
module pour comprendre son mode de fonctionnement.

\vphantom{}

Le deuxième problème, j'ai l'ai eu bien plus tard dans le développement. 
Cependant celui-ci m'a fait perdre pratiquement deux voir trois semaines. Pour 
la mise en place du site web, il avait été convenu lors d'une réunion, 
d'utiliser le paradigme de la \textit{Programmation fonctionnelle réactive}. Ce 
paradigme peut être visualisé par un tableur. Lorsque l'on modifie le contenu 
d'une case, le reste du tableau peut alors évoluer. C'est sur ce principe que 
repose la \textit{programmation réactive}. En \textit{Haskell}, il existe de 
nombreux modules qui permettent la mise en place de cette méthode de 
programmation. 
Cependant, celui le souvent utilisé dans les interfaces Web, \textit{Reflex}, 
nécessite l'outil \textit{Obelisk}. Cet outil m'a tout d'abord causé des 
problèmes quant à son installation, puis son utilisation. L'utilisation de 
\textit{Nix} par celui-ci oblige à chaque installation de compiler au moins 
une fois ce qui a été installé. De ce fait, son utilisation était très longue à 
mettre en \oe uvre. Après cela, il y a aussi l'utilisation du module 
\textit{Reflex}. Bien que ce soit un module Haskell, son utilisation requiert un 
apprentissage similaire à celui d'un langage. Notamment, car 
nous n'avions jamais vu ce paradigme et donc ces mécanismes m'étaient inconnus. 
De ce fait, après avoir passé une semaine à essayer d'apprendre le 
\textit{FRP}, je me suis redirigé vers la mise en place d'une application 
\textit{Gtk} (implémentation bien plus commune, qui permet donc un développement 
plus rapide).  

\vphantom{}

Enfin la dernière difficulté rencontrée à été la rédaction du rapport formel 
dans un premier temps. Bien que j'ai entrepris son écriture en même temps que 
le développement de la bibliothèque. Le formalisme nécessaire et les définitions 
qu'il faut introduire pour que le document se suffise à lui-même a rendu cette 
tâche très chronophage. De plus, n'ayant jamais réellement été confronté à 
cet exercice lors de la licence, il a été dur de s'y adapter. Je me suis 
d'ailleurs rendu compte du nombre de relectures nécessaires pour un tel 
document. Je ne pensais pas qu'il fallait en faire autant. De même pour la 
bibliographie qui doit si elle est citée définir les concepts de la même manière 
que ceux du document.

Cette difficulté de rédaction s'est aussi vue sur la fin de ce stage, lors de la 
rédaction de ce rapport. Ayant maintenant l'habitude des rapports de projet 
grâce à tous ceux rédigés lors de la licence, je ne pensais pas que celui-ci 
serait si compliqué. La différence entre ce rapport et ceux précédents est 
surtout la différence de contenu. L'objectif de celui-ci ne vise pas à montrer 
toutes les facettes du projet sur lequel j'ai travaillé, mais plutôt la manière 
dont j'ai travaillé dessus et ce que j'ai appris. 

