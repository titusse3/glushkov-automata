\section{\textit{Haskell}}

L'\textit{Haskell} est un langage extrêmement différent de ceux vue au cours de 
la licence. Tout d'abord, il s'agit d'un \textbf{langage fonctionnel}. Bien que 
nous ayons vue ce paradigme par le biais du langage \textit{OCaml}, 
nous n’avons jamais été aussi loin. Dans le développement notamment, nous 
n'avons par exemple jamais mis en place de parser ou bien même d'application 
graphique. De plus, le concept de \textbf{programmation pure} est poussé à un 
extrême dans ce langage. Il existe une séparation hermétique entre ce qui est 
\textbf{pure} et ce qu'il ne l'ait pas.

\subsection{Programmation pure}

\begin{quotation}
    \textit{'A pure function is a function that, given the same input, will 
    always return the same output and does not have any observable side effect.'
    }\cite{citationPureProg}
\end{quotation}

Pour revenir sur le concept de \textbf{programmation pure}, dans un langage 
fonctionnelle, on parle alors de \textbf{fonction pure} (Le paradigme, ne 
permettant que la création de fonction). On définit une fonction présente dans 
un programme comme étant \textbf{pure} si et seulement si~:

\begin{itemize}
    \item[\textbullet] La fonction est mathématiquement 
    \textbf{déterministe}. Par cela, nous entendons que pour une fonction \(f\), 
    il ne peut y avoir \(f(x) = y_1\) et \(f(x) = y_2\) avec \(y_1 \neq y_2\). 
    En programmation, on conçoit que de nombreuse fonction 
    sont déterministes tels que le calcule du n-ième terme de la suite de 
    Fibonacci. On peut voir sur la figure~\ref{fig:progFiboHaskell}, une 
    implémentation de cette fonction~(L'implémentation présentée ici, est la 
    version intuitive. Il existe cependant des méthodes bien plus performantes 
    pour se calcule, se référer à l'article de blog~\cite{citationFiboProg}).
    Il vient de manière logique que notre fonction est déterministe
    \begin{figure}[H]
        \begin{minted}{haskell}
            fibo :: Int -> Int
            fibo 0 = 0
            fibo 1 = 1
            fibo n = fibo' 0 1 n
              where
                fibo' _ m2 1  = m2
                fibo' m1 m2 m = fibo' m2 (m1 + m2) $ m - 1
        \end{minted}
        \caption{
            Code \textit{Haskell}, du calcul du n-ième terme de Fibonacci.
        }\label{fig:progFiboHaskell}
    \end{figure}

    \item[\textbullet] La fonction doit ne doit pas produire 
    d'\textbf{effet de bord}. Un \textbf{'side effect'} comme nommé en anglais, 
    est une qualification d'une action qui modifie ou dépend de son environnement 
    extérieur lors d'un calcule. Un exemple très simple peut être trouvé dans 
    l'affichage d'un nombre sur la sortie standard. Dans le langage \textbf{C},
    cet affichage correspondrait au code de la figure~\ref{fig:progAfficheC}.
    Ces \textbf{effets de bord} sont extrêmement problématique, car ils 
    peuvent causer des erreurs non prises en compte par celui qui a conçu la 
    fonction et même par la personne l'utilisant. Dans le cadre d'un affichage, 
    ces erreurs ne sont pas forcément problématiques, mais dans le cas de 
    modification de variable global par exemple, ou même du contenu d'un fichier
    cela pourrait compromettre l'intégrité du code.
    \begin{figure}[H]
        \begin{minted}{c}
            #include <stdio.h>

            void print(int n) {
                printf("%d", n);
            }
        \end{minted}
        \caption{
            Code \textit{C}, d'un affichage sur la sortie standard.
        }\label{fig:progAfficheC}
    \end{figure}
\end{itemize}

\subsection{Les spécificités du langage}

Comme évoquée plus tôt, l'\textit{Haskell} met en place une séparation 
hermétique entre les fonctions pures et impure. C'est d'ailleurs ce qui en fait 
sa plus grande différence à première vue avec l'\textit{OCaml}. Cette séparation
s'effectue avec l'un des nombreux concepts de la \textbf{théorie des catégories}
présente dans ce langage. Les \textbf{Monades} (On ne citera que le terme de 
\textbf{Monade} dans cette partie, mais les structures \textbf{Functor} et 
\textbf{Applicative} visent à la même chose.), est le concept qui permet de 
confiner toute action impure du reste du code. En effet, lors que par exemple 
une action \textit{IO (input/output)} doit être faite, elle se trouvera dans la 
monade \mintinline{haskell}{IO}. Si on reprend la fonction 
\mintinline{haskell}{fibo} de la figure~\ref{fig:progFiboHaskell}, et que l'on 
souhaite afficher le n-ième nombre de Fibonacci avec \mintinline{haskell}{n}, un 
nombre entrée par l'utilisateur on obtient le code de la 
figure~\ref{fig:progInOutFibo}.

\begin{figure}[H]
    \begin{minted}{haskell}
        import Text.Read (readMaybe)
        
        main :: IO ()
        main = do
            l <- getLine
            let n = readMaybe l :: Maybe Int
            print $ fibo <$> n :: IO()
    \end{minted}
    \caption{
        Code \textit{Haskell}, de l'affichage du n-ième nombre de Fibonacci
        entrée par un utilisateur.
    }\label{fig:progInOutFibo}
\end{figure}

Du code ci-dessus, on peut observer le type \mintinline{haskell}{IO}, monade 
d'entrée et de sortie, mais aussi \mintinline{haskell}{Maybe}. Ces deux monades,
sont un bon exemple de ce que concept apporte. On pourrait citer la définition 
donnée par Wikipédia d'une monade figure~\ref{fig:citationMonad}, mais cela ne 
les ferait comprendre qu'à une mince portion de personne.

\begin{figure}[H]
    \begin{quotation}
        \textit{'[a Monad is] an endofunctor, together with two natural 
        transformations required to fulfill certain coherence conditions'
        }\cite{citationMonadWiki}
    \caption{
      Définition d'une Monad selon \textit{Wikipedia}.
    }\label{fig:citationMonad}
    \end{quotation}
\end{figure}

On peut utiliser les monades, comme un outil gèrent les erreurs possibles 
liées aux \textbf{effets de bord}. Ce qui rend ce concept de Monade si 
important est si on dispose d'une fonction 
\mintinline{haskell}{f :: Int -> Int}, et d'une valeur de type 
\mintinline{haskell}{n :: IO (Int)} l'appelle suivant est une erreur levée à la 
compilation \mintinline{haskell}{f n}. Évidemment, il existe des moyens 
d'appeler \mintinline{haskell}{f}, avec la valeur entière contenu dans 
\mintinline{haskell}{n}. Pour ce faire, on doit avoir recours aux fonctions 
suivantes~: 

\begin{minted}{haskell}
  -- Operateur Functor
  fmap :: Functor f => (a -> b) -> f a -> f b
  (<$>) :: Functor f => (a -> b) -> f a -> f b
  -- Operateur Applicative
  (<*>) :: Applicative f => f (a -> b) -> f a -> f b
  -- Operateur Monad
  (>>=) :: Monad m => m a -> (a -> m b) -> m b
  (>>) :: Monad m => m a -> m b -> m b
  return :: Monad m => a -> m a
\end{minted}

On remarque alors que toutes ces fonctions font en quelque sorte en conversion 
vers la monade. Grâce à cette conversion, on peut donc conserver la gestion 
d'erreur mise en place par ces structures.

Qu'une fine partie des concepts de ce langage n'a été abordée ici. Bien d'autre
chose le rende bien différent des langages abordés lors de la licence. Nous 
aurions pu voir par exemple, ce qu'apporte l'évaluation paresseuse du langage. 
Où encore, les optimisations agressives fournies par le compilateur 
\textit{GHC}. De plus, la proximité entre ce langage et la 
\textbf{théorie des catégories}, fait que de nombreux modules introduise des 
notions pas encore implémenter. On peut notamment citée les \textit{Lens}.