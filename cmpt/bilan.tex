\section{Bilan}

\subsection{Ce que j'ai acquis pendant le stage}

Durant ce stage de huit semaines en laboratoire, j'ai acquis de nombreuses 
compétences techniques. J'ai approfondi mes connaissances en 
\textit{programmation fonctionnelle} et notamment en \textit{Haskell}, notamment 
sur la programmation pure et l'utilisation des monades. J'ai développé une 
bibliothèque d'automates, mis en place des tests par propriétés et réalisé des 
benchmarks de performance. Cette expérience m'a également permis de me 
familiariser avec des outils et des modules spécifiques au \textit{Haskell}, 
malgré les défis posés par une documentation qui sort des normes apprises lors 
de la licence. 

\vphantom{}

En termes de bénéfices, j'ai amélioré ma gestion du temps et des priorités grâce 
à l'utilisation de technique de priorisation tel que l'utilisation d'un 
diagramme de \textit{Gantt}. J'ai aussi appris à surmonter les obstacles du 
télétravail, notamment en gérant mes horaires et en résolvant des problèmes 
matériels. En somme, cette expérience a été extrêmement enrichissante, 
confirmant mon intérêt pour l'aspect théorique de l'informatique et m'apportant 
une vision claire des attentes et exigences du secteur, par le biais de la 
rédaction d'une spécification formelle.

\subsection{Les connaissances de la licence qui m'ont été utile}

Les savoirs acquis durant ma licence en informatique m'ont été très utiles tout 
au long de ce stage. Mes connaissances en \textit{programmation fonctionnelle}, 
acquises lors des cours sur \textit{OCaml}, m'ont fourni une base solide pour 
aborder ce nouveau langage. De plus, les compétences en algorithmique et en 
structures de données m'ont aidé à développer et optimiser les algorithmes 
nécessaires pour la bibliothèque d'automates. Bien évidement, la matière de 
théorie des langages m’a aidé quant à la compréhension de notions déjà vue 
auparavant.

% rajouter des choses

\subsection{Affinement du projet professionnel}

