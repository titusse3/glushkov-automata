\section{Bilan}

\subsection{Ce que j'ai acquis pendant le stage}

Durant ce stage de huit semaines en laboratoire, j'ai acquis de nombreuses 
compétences techniques. J'ai approfondi mes connaissances en 
\textit{programmation fonctionnelle} et notamment en \textit{Haskell}, notamment 
sur la programmation pure et l'utilisation des monades. J'ai développé une 
bibliothèque d'automates, mis en place des tests par propriétés et réalisé des 
benchmarks de performance. Cette expérience m'a également permis de me 
familiariser avec des outils et des modules spécifiques au \textit{Haskell}, 
malgré les défis posés par une documentation qui sort des normes apprises lors 
de la licence. 

\vphantom{}

En termes de bénéfices, j'ai amélioré ma gestion du temps et des priorités grâce 
à l'utilisation de technique de priorisation tel que l'utilisation d'un 
diagramme de \textit{Gantt}. J'ai aussi appris à surmonter les obstacles du 
télétravail, notamment en gérant mes horaires et en résolvant des problèmes 
matériels. En somme, cette expérience a été extrêmement enrichissante, 
confirmant mon intérêt pour l'aspect théorique de l'informatique et m'apportant 
une vision claire des attentes et exigences du secteur, par le biais de la 
rédaction d'une spécification formelle.

\subsection{Les connaissances de la licence qui m'ont été utile}

Les savoirs acquis durant ma licence en informatique m'ont été très utiles tout 
au long de ce stage. Mes connaissances en \textit{programmation fonctionnelle}, 
acquises lors des cours sur \textit{OCaml}, m'ont fourni une base solide pour 
aborder ce nouveau langage. De plus, les compétences en algorithmique et en 
structures de données m'ont aidé à développer et optimiser les algorithmes 
nécessaires pour la bibliothèque d'automates. Bien évidement, la matière de 
théorie des langages m’a aidé quant à la compréhension de notions déjà vue 
auparavant.

\vphantom{}

Les projets pratiques réalisés durant ma licence m'ont préparé à aborder des 
possibles problèmes techniques ainsi que la gestion du temps. La rigueur avec 
laquelle on nous à enseigner les langages \textit{C} et \textit{Java}, m’a 
poussé à toujours aller plus loin dans l'apprentissage du \textit{Haskell}. Avec 
l’écure systématique des documentations, mais aussi d'essayer de comprendre au 
maximum ce qu'il se passait. De plus, il se trouve que certain projet que nous 
avons du produire pour la licence se trouve très près de certaine partie de 
celui-ci. Le parser d'expression régulière utilise les modules \textit{Happy} et 
\textit{Alex}, ces deux modules sont basés sur \textit{Bison} et \textit{Lex} ce 
qui a rendu leur utilisation bien plus simple. En effet, lors du projet de 
compilation, nous avions du utilisés \textit{Bison} et \textit{Lex} afin de 
mettre en place un compilateur assez simple. 

\subsection{Affinement du projet professionnel}

Cette expérience m'a définitivement aidé à affiner mon projet professionnel et 
mon projet d'étude. Travailler sur un projet réel en \textit{Haskell} et 
contribuer à une bibliothèque d'automates m'a confirmé mon intérêt pour l'aspect 
théorique de l'informatique. Le langage \textit{Haskell} en est un parfait 
exemple, avec ces bases dans la théorie des catégories. Je prévois d'ailleurs de 
commencer mon apprentissage de ce sujet durant les vacances qui vont suivre ce 
stage.

\vphantom{}

Elle m'a également permis de mieux définir mes objectifs professionnels. Ce 
stage m'a conforté dans l'idée de devenir chercheur. Cette expérience m'a 
conforté dans mon choix de master, celui \textit{ITA}. Travailler sur des 
projets de recherche concrets, analyser les résultats et proposer des 
améliorations m'ont donné un aperçu de la rigueur et de la créativité 
nécessaires dans le domaine de la recherche. Le stage m'a aussi permis 
d'échanger avec un étudiant en 2e année de ce même master, ce qui a enrichi ma 
vision de celui-ci.