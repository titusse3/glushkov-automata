\documentclass[12pt]{article}
\usepackage[utf8]{inputenc}
\usepackage[french]{babel}
\usepackage{graphicx}
\usepackage[export]{adjustbox}
\usepackage[margin=3cm]{cmpt}
\usepackage{float}
\usepackage{amsfonts}
\usepackage{hyperref}
\usepackage{pgfgantt}
\usepackage{hyperref}
\usepackage[backend=biber,style=numeric,sorting=none]{biblatex}
\usepackage{listings}
\usepackage{xcolor}
\usepackage{minted}
\usepackage{caption}
\usepackage{booktabs}
\addbibresource{references.bib}
\captionsetup{justification=centering}

%--- begin document ------------------------------------------------------------

\title{Rapport de stage}
\author{T.Renaux Verdiere}
\date{2023--2024}

\begin{document}

\begin{figure}
    \includegraphics[scale=0.3, right]{logo-univ-rouen-normandie-noir.png}
\end{figure}

\maketitle

\begin{abstract}
    Ce document constitue mon rapport de stage de huit semaines. Ce stage a été
    effectué au laboratoire de recherche de l'université où j'ai effectué cette  
    licence, le GR\textsuperscript{2}IF (Groupe de Recherche Rouennais en 
    Informatique Fondamentale). Il s'agit d'un laboratoire situé sur le campus 
    du Madrillet de l'Université de Rouen, il est composé de dix enseignants 
    chercheurs et de deux personnels administratifs. Fondé en mai 2018, il se 
    concentre sur diverses thématiques telles que les langages formels, la 
    combinatoire algébrique, l'informatique quantique et le génie logiciel.
\end{abstract}

\newpage   

\tableofcontents

\newpage

\section{Introduction}

Ce rapport présente les résultats de mon stage de huit semaines au sein du 
GR\textsuperscript{2}IF (Groupe de Recherche Rouennais en Informatique 
Fondamentale), sous la supervision de M. Mignot. Ce stage m'a permis de 
développer une bibliothèque d'automates en \textit{Haskell}, d'implémenter des 
tests de propriétés, et de réaliser des benchmarks de performance.

Le rapport commence par une introduction qui présente le contexte du stage. Il 
se poursuit avec une description des objectifs du stage, notamment le 
développement de la bibliothèque d'automates et la mise en place d'une interface 
graphique. La section suivante détaille l'organisation du travail, incluant les 
réunions hebdomadaires et la planification à l'aide d'un diagramme de Gantt. Les 
difficultés rencontrées, tant au niveau du stage que des aspects techniques, 
sont ensuite abordées. Une section est dédiée à la présentation du langage 
\textit{Haskell}, certaines spécificités de la programmation pure. Les résultats
du stage sont ensuite exposés, avec une présentation détaillée de la 
bibliothèque d'automates, de l'application graphique, ainsi que des phases de 
tests par propriétés et benchmarks. Les possibilités d'amélioration du projet 
sont ensuite discutées, suivies par un bilan des compétences acquises et de 
l'utilisation des connaissances de la licence. Le rapport se termine par 
l'affinement professionnel que ce stage m'a permis de mettre en place.

\section{Objectif du Stage}

Ce stage a pour objectif, le développement d'une bibliothèque traitant 
d'automates à partir d'expression rationnelle en \textit{Haskell}. De plus, 
cette bibliothèque doit être accompagnée d'une description formelle. Que ce soit 
des algorithmes et même des structures utilisées. Il a de plus été envisagé 
assez tôt dans ce stage, la mise en place d'une interface graphique qui pourrait 
améliorez l'accessibilité de cette bibliothèque.

\vphantom{}

La bibliothèque d'automate ne devait pas seulement permettre de créer un 
automate, elle devait aussi mettre en place des fonctions usuelles sur les 
automates. Toutes ces fonctions sont détaillées dans la spécification technique 
de la bibliothèque. De même, le développement d'un parser de chaine de caractère 
vers une expression régulière et d'un fichier \textit{JSON} vers un automate 
devait être mis en place.

\vphantom{}

Pour améliorer la lisibilité, une représentation graphique à l'aide du langage 
\textit{dot} et de ces utilitaires devait être mis en place. Enfin, un site 
web avait était prévue pour mettre en place une interface graphique améliorant 
l'accessibilité de toutes les fonctionnalités citées précédemment. 

\subsection{Organisation du Travail}

Le développement de cette bibliothèque se fait parallèlement en \textit{Haskell}
et en \textit{Rust}. En effet, nous sommes un binôme sur ce projet. Bien que les 
objectifs soient les mêmes, les implémentations n'ont rien à voir. La différence
entre les deux langages est si grande que très peu de ressemblance peuvent être 
trouvé.

\vphantom{}

L'organisation du travail pour ce stage, a été mis en place lors de notre 
première réunion qui s'est tenu le premier jour de ce stage. Il a alors été 
convenu de mettre en place une réunion hebdomadaire. De ce fait, toutes les 
semaines, une réunion pour parler des avancées, difficultés et objectif ont été 
mises en place. De manière générale, de telle réunion avait un temps variable 
pouvant aller de 20 minute à pour les plus longs plus d'une heure.

\vphantom{}

J'ai pour ma part après cette première réunion qui nous as donnés les objectifs
de ce stage, mis en place un diagramme de \textit{Gantt}, celui de la 
figure~\ref{fig:gantfig}. Comme on peut le voir sur cette figure, chaque couleur 
corresponde à une tâche en particulier. Ce diagramme aillant été fait au début 
de mon stage, certain objectif ont pu prendre plus de temps que prévue. La 
correspondance couleur, objectifs est celle ci-dessous~:

\vphantom{}

\begin{itemize}
  \item[\textbullet] \textcolor{red}{Rouge}~: Mise en place d'un type 
  d'\textbf{expression régulière}, ainsi que d'\textbf{automates}. Puis 
  implémenter la conversion d'expression vers automate à l'aide de l'algorithme 
  de \textit{Glushkov}.
  \item[\textbullet] \textcolor{blue}{Bleu}~: Développement d'un \textbf{Lexer}, 
  aillant pour but de convertir une chaine de caractère en \textbf{expression 
  régulière}.
  \item[\textbullet] \textcolor{green}{Vert}~: Développement de diverses 
  fonctions sur des automates. Que ce soit des \textit{`getters'/ `setters'} ou 
  encore des tests de propriétés comme l'homogénéisée ou encore savoir s'il 
  est standard (se référer au document technique).
  \item[\textbullet] \textcolor{yellow}{Jaune}~: Création d'une conversion d'un
  automate en \textit{Dot}. Pour une représentation graphique qui as pour d'être 
  d'être, elle aussi, développer.
  \item[\textbullet] \textcolor{purple}{Violet}~: Mise en place du site web qui 
  devait permettre d'obtenir une représentation graphique et une meilleure 
  accessibilité.
  \item[\textbullet] \textcolor{orange}{Orange}~: Enfin, ces deux dernières 
  semaines ont été laissées pour peaufiner chacune des parties précédentes, mais 
  aussi de rédiger ce rapport de stage.
\end{itemize}

\begin{figure}[H]
  \begin{ganttchart}[
    hgrid,
    vgrid,
    x unit=2.5mm,
    time slot format=isodate
    ]{2024-04-22}{2024-06-22}
    \gantttitlecalendar{month} \\
    \ganttbar[bar/.append style={fill=red}]{}{2024-04-22}{2024-04-26} 
    \ganttbar[bar/.append style={fill=red}]{}{2024-04-29}{2024-05-03} 
    \ganttbar[bar/.append style={fill=blue}]{}{2024-05-06}{2024-05-07} 
    \ganttbar[bar/.append style={fill=green}]{}{2024-05-13}{2024-05-17} 
    \ganttbar[bar/.append style={fill=green}]{}{2024-05-21}{2024-05-22} 
    \ganttbar[bar/.append style={fill=yellow}]{}{2024-05-23}{2024-05-24}
    \ganttbar[bar/.append style={fill=yellow}]{}{2024-05-27}{2024-05-31} 
    \ganttbar[bar/.append style={fill=purple}]{}{2024-06-03}{2024-06-07} 
    \ganttbar[bar/.append style={fill=orange}]{}{2024-06-10}{2024-06-14}
    \ganttbar[bar/.append style={fill=orange}]{}{2024-06-17}{2024-06-21} \\
    \ganttmilestone[milestone/.append style={shape=rectangle, fill=black}]{}{2024-04-22}
    \ganttmilestone[milestone/.append style={shape=rectangle, fill=black}]{}{2024-04-26}
    \ganttmilestone[milestone/.append style={shape=rectangle, fill=black}]{}{2024-05-02}
    \ganttmilestone[milestone/.append style={shape=rectangle, fill=black}]{}{2024-05-14}
    \ganttmilestone[milestone/.append style={shape=rectangle, fill=black}]{}{2024-05-23}
    \ganttmilestone[milestone/.append style={shape=rectangle, fill=black}]{}{2024-06-06}
    \ganttmilestone[milestone/.append style={shape=rectangle, fill=black}]{}{2024-06-13}
  \end{ganttchart}
  \caption{
    Diagramme de \textit{Gantt} mise en place pour le stage. Les rectangles 
    noirs correspondent au jour de réunion.
  }\label{fig:gantfig}
\end{figure}

% peut être encore parler de quelque chose mais je sais pas quoi

\subsection{Difficultés rencontrées}

J'ai pu au cours de ce stage, rencontrées de nombreuses difficultés. Pour les 
exposées, j'ai décidé de diviser en deux catégories. Les difficultés liées au 
développement, à la partie théorique et les problématiques spécifique d'un 
stage.

\subsubsection{Problème lié au stage}

Ce stage constitue ma première expérience proche d'un travail en entreprise. De 
ce fait, j'ai pu rencontrer les problèmes que je pense nombreux ont. Notamment 
pendent quelque jour, notamment la première semaine, nous étions en télétravail.
De nombreux problèmes survienne par ce mode de travail. Tous d'abord la façon de 
gérer ces horaires. M. Mignot nous as laissé gérer nos horaires nous même ce qui 
à fait en sorte que pendant le télétravail j'ai eu du mal à ne pas trop 
travailler et même dans l'autre sens parfois. Ensuite, on a les problèmes du 
quotidien comme des pertes de connexions qui rendent certaine journée 
pratiquement impossible à avancée. Enfin, je n'ai pas eu de chance lors de ces 
derniers mois et je suis tombé malade, ce qui a accentué encore plus ce 
télétravail.

\newpage

Un autre problème et survenue, celui-ci concernait le matériel. En effet, mon 
ordinateur portable n'avait selon mon système d'exploitation plus de mémoire 
vive. J'ai alors dù réinstaller un nouvel os. Cependant, je n'avais pas prévue 
qu'il existait encore des os qui ne se lancé pas sur \textit{systemd}. Sans 
rentrer dans les détailles, ce problème m’a pris énormément de temps à régler. 
C'est un problème, car le gestionnaire de paquet (s'il peut l'appeler comme ça)
\textit{Nix} (qui est l'outil le plus utilisé pour la création de site en 
\textit{Haskell} sous le paradigme de la \textit{Programmation fonctionnelle 
réactive}) doit ce lancé à l'aide de \textit{systemd}. Sait d'ailleurs 
une des raisons aillant poussé vers la migration du développement d'un site web 
vers une application \textit{Gtk}. Se référer à la sous-section suivante et à la 
partie résultat de ce compte rendu. 

\subsubsection{Difficultés de développement du projet}

Avant tout, j'ai n'ai commencé à apprendre ce langage que depuis décembre. 
Dès lors que j'ai obtenu ce stage et que j'ai appris que le sujet nécessite le 
\textit{Haskell}. J'ai donc demandé des ressources à Mr Mignot et j'ai suivi 
ces documentations. Notamment la lecture des livres 
\textit{Get Programming with Haskell} et \textit{Haskell in Depth}
\cite{bookWithHaskell, haskellInDepth}. Pour combler celà, j'ai aussi suivie 
certain cours vidéo disponible sur \textit{Youtube}
\cite{playHaksell1, playHaksell2}. De ce fait, malgré avoir commencé assez 
tôt l'apprentissage de ce langage, comme évoquer dans de nombreuse ressource la 
courbe d'apprentissage du \textit{Haskell} est exponentielle. Donc de nombreux 
problèmes que je vais évoquer ci-dessous aurait pu ne pas avoir vue le jour, si 
j'avais une plus grande expérience avec ce langage.

\vphantom{}

Le premier problème a été rencontré lors de la mise du convertisseur de chaine 
de caractère vers expression. Les modules utilisés ont été comme conseille par 
M. Mignot, \textit{Alex} et \textit{Happy}. Le problème a été la documentation 
de ces modules. Cette manière de documentation basée uniquement sur ce que font 
chaque fonction et non de comment on peut les utiliser et la non-présence 
d'exemple a rendu ce développement bien plus long. Ce même problème a été vu 
lors de l'utilisation des modules \textit{GraphViz} qui permet la représentation
d'un graphe. La documentation est totalement horrible. De plus, le module étant 
très peu utilisé il n'existe pratiquement aucun exemple sur internet. J'ai donc
du programmer a taton et me référer au code source du module pour comprendre 
son mode de fonctionnement.

\vphantom{}

Le deuxième problème, j'ai l'ai eu bien plus tard dans le développement. 
Cependant celui-ci m’a fait perdre pratiquement deux voir trois semaines. Pour 
la mise en place site web, il avait été convenu lors d'une réunion, d'utilisé 
le paradigme de la \textit{Programmation fonctionnelle réactive}. 
% finir cette partie

\vphantom{}

Enfin la dernière difficulté rencontrée à été la rédaction du rapport formelle 
dans un premier temps. Bien que j'aie entrepris son écriture en même temps que 
le développement de la bibliothèque. Le formalisme nécessaire et les définitions 
qu'il faut introduire pour que le document se suffise à lui-même a rendu cette 
tâche très chronophage. De plus, n'aillant jamais réellement été confronté a 
cet exercice lors de la licence, il a été dur de s'y adapté. Je me suis 
d'ailleurs rendu compte du nombre de relectures nécessaire pour un tel document. 
Je ne pensais pas qu'il fallait en faire au temps. De même pour la bibliographie 
qui doit si elle est citée définir les concepts de la même manière que ceux 
du document.

Cette difficulté de rédaction s'est aussi vue sur la fin de ce stage, lors de la 
rédaction de ce rapport. Aillant maintenant l'habitude des rapports de projet 
avec tous ceux rédiger lors de la licence, je ne pensais pas que celui-ci aller 
être si dure. La différence entre ce rapport et ceux précédent est surtout la 
différence de contenu. L'objectif de celui-ci ne vise pas à montrer toutes les 
facettes du projet sur laquelle j'ai travaillé, mais plutôt la manière dont j'ai 
travaillé déçu est ce que j'ai appris. 



\section{\textit{Haskell}}

\textit{Haskell} est un langage extrêmement différent de ceux vue au cours de 
la licence. Tout d'abord, il s'agit d'un \textbf{langage fonctionnel}. Bien que 
nous ayons vu ce paradigme par le biais du langage \textit{OCaml}, 
nous n’avons jamais été aussi loin. Dans le développement notamment, nous 
n'avons par exemple jamais mis en place de parser ou bien même d'application 
graphique. De plus, le concept de \textbf{programmation pure} est poussé à un 
extrême dans ce langage. Il existe une séparation hermétique entre ce qui est 
\textbf{pure} et ce qu'il ne l'est pas.

\subsection{Programmation pure}

\begin{quotation}
    \textit{'A pure function is a function that, given the same input, will 
    always return the same output and does not have any observable side effect.'
    }\cite{citationPureProg}
\end{quotation}

Pour revenir sur le concept de \textbf{programmation pure}, dans un langage 
fonctionnelle, on parle alors de \textbf{fonction pure} (Le paradigme, ne 
permettant que la création de fonction). On définit une fonction présente dans 
un programme comme étant \textbf{pure} si et seulement si~:

\begin{itemize}
    \item[\textbullet] La fonction est mathématiquement 
    \textbf{déterministe}. Par cela, nous entendons que pour une fonction \(f\), 
    il ne peut y avoir \(f(x) = y_1\) et \(f(x) = y_2\) avec \(y_1 \neq y_2\). 
    En programmation, on conçoit que de nombreuse fonction 
    sont déterministes tels que le calcule du n-ième terme de la suite de 
    Fibonacci. On peut voir sur la figure~\ref{fig:progFiboHaskell}, une 
    implémentation de cette fonction~(l'implémentation présentée ici, est la 
    version intuitive. Il existe cependant des méthodes bien plus performantes 
    pour se calcule, se référer à l'article de blog~\cite{citationFiboProg}).
    Il vient de manière logique que notre fonction est déterministe
    \begin{figure}[H]
        \begin{minted}{haskell}
            fibo :: Int -> Int
            fibo 0 = 0
            fibo 1 = 1
            fibo n = fibo' 0 1 n
              where
                fibo' _ m2 1  = m2
                fibo' m1 m2 m = fibo' m2 (m1 + m2) $ m - 1
        \end{minted}
        \caption{
            Code \textit{Haskell}, du calcul du n-ième terme de Fibonacci.
        }\label{fig:progFiboHaskell}
    \end{figure}

    \item[\textbullet] La fonction ne doit pas produire 
    d'\textbf{effet de bord}. Un \textbf{'side effect'} comme nommé en anglais, 
    est une qualification d'une action qui modifie ou dépend de son environnement 
    extérieur lors d'un calcule. Un exemple très simple peut être trouvé dans 
    l'affichage d'un nombre sur la sortie standard. Dans le langage \textbf{C},
    cet affichage correspondrait au code de la figure~\ref{fig:progAfficheC}.
    Ces \textbf{effets de bord} sont extrêmement problématique, car ils 
    peuvent causer des erreurs non prises en compte par celui qui a conçu la 
    fonction et même par la personne l'utilisant. Dans le cadre d'un affichage, 
    ces erreurs ne sont pas forcément problématiques, mais dans le cas de 
    modification de variable global par exemple, ou même du contenu d'un fichier
    cela pourrait compromettre l'intégrité du code.
    \begin{figure}[H]
        \begin{minted}{c}
            #include <stdio.h>

            void print(int n) {
                printf("%d", n);
            }
        \end{minted}
        \caption{
            Code \textit{C}, d'un affichage sur la sortie standard.
        }\label{fig:progAfficheC}
    \end{figure}
\end{itemize}

\subsection{Les spécificités du langage}

Comme évoqué plus tôt, \textit{Haskell} met en place une séparation 
hermétique entre les fonctions pures et impures. C'est d'ailleurs ce qui en fait 
sa plus grande différence à première vue avec \textit{OCaml}. Cette séparation
s'effectue avec l'un des nombreux concepts de la \textbf{théorie des catégories}
présente dans ce langage. Les \textbf{Monades} (On ne citera que le terme de 
\textbf{Monade} dans cette partie, mais les structures \textbf{Functor} et 
\textbf{Applicative} visent à la même chose.), est le concept qui permet de 
confiner toute action impure du reste du code. Par exemple 
une action \textit{IO (input/output)} doit être faite, elle se trouvera dans la 
monade \mintinline{haskell}{IO}. Si on reprend la fonction 
\mintinline{haskell}{fibo} de la figure~\ref{fig:progFiboHaskell}, et que l'on 
souhaite afficher le n-ième nombre de Fibonacci avec \mintinline{haskell}{n}, un 
nombre entrée par l'utilisateur on obtient le code de la 
figure~\ref{fig:progInOutFibo}.

\begin{figure}[H]
    \begin{minted}{haskell}
        import Text.Read (readMaybe)
        
        main :: IO ()
        main = do
            l <- getLine
            let n = readMaybe l :: Maybe Int
            print $ fibo <$> n :: IO()
    \end{minted}
    \caption{
        Code \textit{Haskell}, de l'affichage du n-ième nombre de Fibonacci
        entrée par un utilisateur.
    }\label{fig:progInOutFibo}
\end{figure}

Du code ci-dessus, on peut observer le type \mintinline{haskell}{IO}, monade 
d'entrée et de sortie, mais aussi \mintinline{haskell}{Maybe}. Ces deux monades,
sont un bon exemple de ce que concept apporte. On pourrait citer la définition 
donnée par Wikipédia d'une monade figure~\ref{fig:citationMonad}, mais cela ne 
les ferait comprendre qu'à une mince portion de personne.

\begin{figure}[H]
    \begin{quotation}
        \textit{'[a Monad is] an endofunctor, together with two natural 
        transformations required to fulfill certain coherence conditions'
        }\cite{citationMonadWiki}
    \caption{
      Définition d'une Monad selon \textit{Wikipedia}.
    }\label{fig:citationMonad}
    \end{quotation}
\end{figure}

On peut utiliser les monades, comme un outil gerant les erreurs possibles 
liées aux \textbf{effets de bord}. Ce qui rend ce concept de Monade si 
important est si on dispose d'une fonction 
\mintinline{haskell}{f :: Int -> Int}, et d'une valeur de type 
\mintinline{haskell}{n :: IO (Int)} l'appelle suivant est une erreur levée à la 
compilation \mintinline{haskell}{f n}. Évidemment, il existe des moyens 
d'appeler \mintinline{haskell}{f}, avec la valeur entière contenue dans 
\mintinline{haskell}{n}. Pour ce faire, on doit avoir recours aux fonctions 
suivantes~: 

\begin{minted}{haskell}
  -- Operateur Functor
  fmap :: Functor f => (a -> b) -> f a -> f b
  (<$>) :: Functor f => (a -> b) -> f a -> f b
  -- Operateur Applicative
  (<*>) :: Applicative f => f (a -> b) -> f a -> f b
  -- Operateur Monad
  (>>=) :: Monad m => m a -> (a -> m b) -> m b
  (>>) :: Monad m => m a -> m b -> m b
  return :: Monad m => a -> m a
\end{minted}

On remarque alors que toutes ces fonctions font en quelque sorte une conversion 
vers la monade. Grâce à cette conversion, on peut donc conserver la gestion 
d'erreur mise en place par ces structures.

Qu'une fine partie des concepts de ce langage n'a été abordée ici. Bien d'autres
choses le rende bien différent des langages abordés lors de la licence. Nous 
aurions pû voir par exemple, ce qu'apporte l'évaluation paresseuse du langage. 
Où encore, les optimisations agressives fournies par le compilateur 
\textit{GHC}. De plus, la proximité entre ce langage et la 
\textbf{théorie des catégories}, fait que de nombreux modules introduisent des 
notions pas encore implémentées. On peut notamment citée les \textit{Lens}.

\section{Résultat}

\subsection{La bibliothèque d'automate}

\subsection{L'application graphique}

\subsection{Phases de tests}

Pour ce projet, la phase de test, je l'ai mis en place à la fin de ce stage. En 
effet, cette partie n'avait été prévue d'être réalisé que s'il restait assez de 
temps pour ce faire. Les réunions hebdomadaires servaient entre autre à jouer en 
partie ce rôle. De plus à chaque ajout de fonctionnalités, j'essayai d'effectuer 
certain teste. Malgré cela, j'ai pu trouver le temps de mettre en place deux 
types de tests. Nous commencerons par voir la façon dont j'ai mis en place une 
des tests par propriétés. Puis nous verrons les testes de performances mises en 
place entre les deux implémentations d'automate.

\subsubsection{Test par propriétés}

% explication de comment fonctionne les testes propriétés
Les tests par propriétés, \textit{`property testing'} en anglais. Est une 
méthode de tests logicielle qui visent à tester les propriétés d'un résultat et 
non pas à comparer le résultat obtenu avec celui espérer~\cite{propertyTesting}. 
De plus, cette méthode ne nécessite pas d'avoir une autre implémentation pour 
comparer les résultats ou même de calculer à la main les résultats des 
opérations que l'on veut tester (Cette méthode est celle communément utilisé, 
des \textit{tests unitaires}). 

\vphantom{}

Outre cela, cette méthode de test convient extrêmement bien à la production 
d'objet mathématique. Notamment en ce qui concerne les automates, ils existent 
de nombreuses propriétés connues qui permet de facilité ces testées. Plus 
précisément, on peut par exemple pour les \textit{automates de Glushkov} se 
référer à l'article \textit{Characterization of Glushkov Automata} de Pascal 
Caron et Djelloul Ziadi~\cite{CaronZiadi2001}. Bien évidemment, cette méthode ne
montre pas que le programme fait exactement ce qu'il doit faire. Il faudrait 
pour cela effectue des preuves formelles mathématiques, tel que celle vue en 
cours d'algorithmique lors de cette licence (par le biais de la logique de 
\textit{Hoare} par exemple). Seulement ces preuves se révélant très longue à 
écrire, les tests par propriétés suffisent donc.  

\vphantom{}

% La façon dont ces mis en place en haskell
En \textit{Haskell}, c'est la bibliothèque \textit{QuickCheck} qui permet de 
mettre en place ce type de test. Une fois des propriétés définie, une génération 
aléatoire de cas de test sera effectuer, et ça sera sur ces cas que les 
propriétés se verrons testé. Pour générer ces cas de test aléatoire, il faut 
que les objets en question dérive la classe \mintinline{haskell}{Arbitrary} du 
module. Le terme `dérivé' en signifie ici que notre classe doit implémenter la 
méthode \mintinline{haskell}{arbitrary :: Gen a} avec \mintinline{haskell}{a}, 
le type que l'on veut tester. C'est grâce à cette fonction 
\mintinline{haskell}{arbitrary} que seront généré de façon aléatoire les objets 
sur lequel seront testé les propriétés. Enfin, pour tester ces propriétés, il 
faut appeler la fonction \mintinline{haskell}{quickCheck} qui prendra en 
paramètre une fonction qui doit avoir ce type 
\mintinline{haskell}{prop :: Arbitrary a => a -> Bool}. Cette fonction 
effectuera une génération de 100 objets et testera la fonction 
\mintinline{haskell}{prop} sur ces objets.

\vphantom{}

% la façon dont j'ai tester les propriétés et le réulstat de celui-ci


\subsubsection{Benchmark des types d'automates}

% Comment fonctionne les benchmarks en haskell

\vphantom{}

% les résultats obtenu sur les temps

% peut être plus tard parler des perfs en mémoires...

\subsection{Les possibilités d'amélioration}


\section{Bilan}

\subsection{Ce que j'ai acquis pendant le stage}

Durant ce stage de huit semaines en laboratoire, j'ai acquis de nombreuses 
compétences techniques. J'ai approfondi mes connaissances en 
\textit{programmation fonctionnelle} et notamment en \textit{Haskell}, notamment 
sur la programmation pure et l'utilisation des monades. J'ai développé une 
bibliothèque d'automates, mis en place des tests par propriétés et réalisé des 
benchmarks de performance. Cette expérience m'a également permis de me 
familiariser avec des outils et des modules spécifiques au \textit{Haskell}, 
malgré les défis posés par une documentation qui sort des normes apprises lors 
de la licence. 

\vphantom{}

En termes de bénéfices, j'ai amélioré ma gestion du temps et des priorités grâce 
à l'utilisation de technique de priorisation tel que l'utilisation d'un 
diagramme de \textit{Gantt}. J'ai aussi appris à surmonter les obstacles du 
télétravail, notamment en gérant mes horaires et en résolvant des problèmes 
matériels. En somme, cette expérience a été extrêmement enrichissante, 
confirmant mon intérêt pour l'aspect théorique de l'informatique et m'apportant 
une vision claire des attentes et exigences du secteur, par le biais de la 
rédaction d'une spécification formelle.

\subsection{Les connaissances de la licence qui m'ont été utile}

Les savoirs acquis durant ma licence en informatique m'ont été très utiles tout 
au long de ce stage. Mes connaissances en \textit{programmation fonctionnelle}, 
acquises lors des cours sur \textit{OCaml}, m'ont fourni une base solide pour 
aborder ce nouveau langage. De plus, les compétences en algorithmique et en 
structures de données m'ont aidé à développer et optimiser les algorithmes 
nécessaires pour la bibliothèque d'automates. Bien évidement, la matière de 
théorie des langages m’a aidé quant à la compréhension de notions déjà vue 
auparavant.

\vphantom{}

Les projets pratiques réalisés durant ma licence m'ont préparé à aborder des 
possibles problèmes techniques ainsi que la gestion du temps. La rigueur avec 
laquelle on nous à enseigner les langages \textit{C} et \textit{Java}, m’a 
poussé à toujours aller plus loin dans l'apprentissage du \textit{Haskell}. Avec 
l’écure systématique des documentations, mais aussi d'essayer de comprendre au 
maximum ce qu'il se passait. De plus, il se trouve que certain projet que nous 
avons du produire pour la licence se trouve très près de certaine partie de 
celui-ci. Le parser d'expression régulière utilise les modules \textit{Happy} et 
\textit{Alex}, ces deux modules sont basés sur \textit{Bison} et \textit{Lex} ce 
qui a rendu leur utilisation bien plus simple. En effet, lors du projet de 
compilation, nous avions du utilisés \textit{Bison} et \textit{Lex} afin de 
mettre en place un compilateur assez simple. 

\subsection{Affinement du projet professionnel}

Cette expérience m'a définitivement aidé à affiner mon projet professionnel et 
mon projet d'étude. Travailler sur un projet réel en \textit{Haskell} et 
contribuer à une bibliothèque d'automates m'a confirmé mon intérêt pour l'aspect 
théorique de l'informatique. Le langage \textit{Haskell} en est un parfait 
exemple, avec ces bases dans la théorie des catégories. Je prévois d'ailleurs de 
commencer mon apprentissage de ce sujet durant les vacances qui vont suivre ce 
stage.

\vphantom{}

Elle m'a également permis de mieux définir mes objectifs professionnels. Ce 
stage m'a conforté dans l'idée de devenir chercheur. Cette expérience m'a 
conforté dans mon choix de master, celui \textit{ITA}. Travailler sur des 
projets de recherche concrets, analyser les résultats et proposer des 
améliorations m'ont donné un aperçu de la rigueur et de la créativité 
nécessaires dans le domaine de la recherche. Le stage m'a aussi permis 
d'échanger avec un étudiant en 2e année de ce même master, ce qui a enrichi ma 
vision de celui-ci.

\newpage

\printbibliography

\end{document}