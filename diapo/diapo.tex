%----------------------------------------------------------------------------------------






%	PACKAGES AND THEMES
%----------------------------------------------------------------------------------------
\documentclass[aspectratio=169,xcolor=dvipsnames]{beamer}
\usetheme{SimplePlus}

\usepackage{hyperref}
\usepackage{graphicx} % Allows including images
\usepackage{booktabs} % Allows the use of \toprule, \midrule and \bottomrule in tables
\usepackage[french]{babel}
\usepackage{pgfgantt}
\usepackage[absolute,overlay]{textpos}
\usepackage{calc}
\usepackage{minted}
\usepackage{svg}
\usepackage{subcaption}

%----------------------------------------------------------------------------------------
%	TITLE PAGE
%----------------------------------------------------------------------------------------

\title[short title]{Stage en laboratoire de recherche}
\subtitle{
    Mise en place d'une bibliothèque traitant des automates\\ 
    en \textit{Rust} et en \textit{Haskell}
}

\author[E. HADDAG, T. RENAUX VERDIERE] {E. HADDAG, T. RENAUX VERDIERE}

\institute[Université de Rouen] 
{
    Université de Rouen Normandie \\
    UFR Sciences \& Techniques, campus du Madrillet
}
\date{\today} % Date, can be changed to a custom date

% PLAN : 
% - présentation du labo/ du sujet du stage
%   * présentation du labo (un projet suqr lequel il ont travailler, avec un lien)
%   * sujet du stage
%   * Organisation (diag de gantt)
% - présentation d'un peu de tdl:
%   * Exp
%   * automate
%   * autoamte de glushkov
%   * orbites
%   * prop orbites
%   * prop orbites de glushkov
% - Implémentation
%   * Rust
%   * Haskell
% - small benchmakr de nos deux applis 
% - Démonstration application
%   * Rust
%   * Haskell
% - Bilan
%   * Comment la licence nous a aidé pendant ce stage
%   * Ce que sa nous a apporté
%   * affinment projet pro
% - Fin

\begin{document}

%------------------------------------------------

\begin{frame}
    \titlepage
\end{frame}

%------------------------------------------------

\begin{frame}{Contexte du stage}
    \begin{figure}
        \centering
        \begin{minipage}{0.45\textwidth}
            \includegraphics[width=\textwidth]{logo_GR2IF.png}
            \caption{Logo du laboratoire}
            \centering
        \end{minipage}
        \hfill
        \begin{minipage}{0.45\textwidth}
            \begin{itemize}
                \item Composer des dix enseignants chercheurs et de deux personnels administratif
                \item Ces domaines~:
                    \begin{itemize}
                        \item Informatique Quantique
                        \item Théorie des langages
                        \item Combinatoire 
                        \item Génie logiciel
                    \end{itemize}
            \end{itemize}
        \end{minipage}
    \end{figure}
\end{frame}

%------------------------------------------------

\begin{frame}
    \begin{block}{Sujet du stage}
        Étude de la conversion d'expression rationnelle en automates de 
        \textit{Glushkov}.
        Production d'une bibliothèque \textit{Haskell} et \textit{Rust} de 
        manipulation de cette conversion.    
    \end{block}
\end{frame}

%------------------------------------------------

\begin{frame}
    \begin{figure}        
        \begin{ganttchart}[
            hgrid,
            vgrid,
            expand chart=\linewidth,
            time slot format=isodate
            ]{2024-04-22}{2024-06-22}
            \gantttitlecalendar{month} \\
            \ganttbar[bar/.append style={fill=red}]{}{2024-04-22}{2024-04-26} 
            \ganttbar[bar/.append style={fill=red}]{}{2024-04-29}{2024-05-03} 
            \ganttbar[bar/.append style={fill=blue}]{}{2024-05-06}{2024-05-07} 
            \ganttbar[bar/.append style={fill=green}]{}{2024-05-13}{2024-05-17} 
            \ganttbar[bar/.append style={fill=green}]{}{2024-05-21}{2024-05-22} 
            \ganttbar[bar/.append style={fill=yellow}]{}{2024-05-23}{2024-05-24}
            \ganttbar[bar/.append style={fill=yellow}]{}{2024-05-27}{2024-05-31} 
            \ganttbar[bar/.append style={fill=purple}]{}{2024-06-03}{2024-06-07} 
            \ganttbar[bar/.append style={fill=orange}]{}{2024-06-10}{2024-06-14}
            \ganttbar[bar/.append style={fill=orange}]{}{2024-06-17}{2024-06-21} \\
            \ganttmilestone[milestone/.append style={shape=rectangle, fill=black}]{}{2024-04-22}
            \ganttmilestone[milestone/.append style={shape=rectangle, fill=black}]{}{2024-04-26}
            \ganttmilestone[milestone/.append style={shape=rectangle, fill=black}]{}{2024-05-02}
            \ganttmilestone[milestone/.append style={shape=rectangle, fill=black}]{}{2024-05-14}
            \ganttmilestone[milestone/.append style={shape=rectangle, fill=black}]{}{2024-05-23}
            \ganttmilestone[milestone/.append style={shape=rectangle, fill=black}]{}{2024-06-06}
            \ganttmilestone[milestone/.append style={shape=rectangle, fill=black}]{}{2024-06-13}
            \ganttmilestone[milestone/.append style={shape=rectangle, fill=black}]{}{2024-06-18}
            \ganttmilestone[milestone/.append style={shape=rectangle, fill=black}]{}{2024-06-21} \\

            \ganttbar[bar/.append style={fill=blue}]{}{2024-04-22}{2024-04-26}
            \ganttbar[bar/.append style={fill=blue}]{}{2024-04-29}{2024-05-03}
            \ganttbar[bar/.append style={fill=blue}]{}{2024-05-06}{2024-05-07}
            \ganttbar[bar/.append style={fill=blue}]{}{2024-05-13}{2024-05-14}
            \ganttbar[bar/.append style={fill=blue}]{}{2024-05-16}{2024-05-17}
            \ganttbar[bar/.append style={fill=blue}]{}{2024-05-21}{2024-05-24}
            \ganttbar[bar/.append style={fill=blue}]{}{2024-05-27}{2024-05-31}
            \ganttbar[bar/.append style={fill=blue}]{}{2024-06-11}{2024-06-11}
            \ganttbar[bar/.append style={fill=blue}]{}{2024-06-13}{2024-06-13} \\

            \ganttbar[bar/.append style={fill=pink}]{}{2024-05-14}{2024-05-17}
            \ganttbar[bar/.append style={fill=pink}]{}{2024-05-21}{2024-05-24}
            \ganttbar[bar/.append style={fill=pink}]{}{2024-06-03}{2024-06-03}
            \ganttbar[bar/.append style={fill=pink}]{}{2024-06-05}{2024-06-07}
            \ganttbar[bar/.append style={fill=pink}]{}{2024-06-12}{2024-06-13} \\

            \ganttbar[bar/.append style={fill=purple}]{}{2024-05-03}{2024-05-03}
            \ganttbar[bar/.append style={fill=purple}]{}{2024-05-06}{2024-05-07}
            \ganttbar[bar/.append style={fill=purple}]{}{2024-05-13}{2024-05-13}
            \ganttbar[bar/.append style={fill=purple}]{}{2024-05-16}{2024-05-17}
            \ganttbar[bar/.append style={fill=purple}]{}{2024-05-23}{2024-05-24}
            \ganttbar[bar/.append style={fill=purple}]{}{2024-06-03}{2024-06-05}
            \ganttbar[bar/.append style={fill=purple}]{}{2024-06-07}{2024-06-08}
            \ganttbar[bar/.append style={fill=purple}]{}{2024-06-10}{2024-06-14}
        \end{ganttchart}
        \caption{Diagramme de \textit{Gantt}.}
    \end{figure}
\end{frame}

%------------------------------------------------

\begin{frame}{Mise en contexte}
    \begin{block}{Expression régulière}
        Si \(E\) est une \textbf{expression régulière} sur \(\Sigma\) (un 
        alphabet) elle peut être égale à~:
        \begin{itemize}
            \item \(E = \varnothing\)
            \item \(E = \varepsilon\)
            \item \(E = a \in \Sigma\)
            \item \(E = F + G\)
            \item \(E = F . G\)
            \item \(E = F^*\)
        \end{itemize}
        Avec \(F, G\) deux expressions régulières sur l'alphabet \(\Sigma\).
    \end{block}
    \begin{examples}{Exemples}
        \(E = (a + b)^*.\varepsilon + \varnothing\)
    \end{examples}
\end{frame}

%------------------------------------------------

\begin{frame}
    \begin{block}{Automate}
        Un \textbf{automate} est un 5-uplet \((\Sigma, Q, I, F, \delta)\) qui 
        peut être représenté graphiquement de la sorte~: 
    \end{block}
    \begin{figure}
        \includegraphics[width=\linewidth]{automate_haskell.png}
    \end{figure}
\end{frame}

%------------------------------------------------

\begin{frame}
    \begin{block}{Propriétés d'automates}
        Certaine propriété des automates définie par M.Caron et M.Ziadi~:
        \begin{itemize}
            \item une \textbf{orbite}
            \item une \textbf{orbite maximale}
        \end{itemize}
    \end{block}
    \begin{figure}
        \includegraphics[width=\linewidth]{exemple_orbite.png}
    \end{figure}
\end{frame}

%------------------------------------------------

\begin{frame}
    \begin{block}{Propriétés d'automates}
        Certaine propriété des automates définie par M.Caron et M.Ziadi~:
        \begin{itemize}
            \item les \textbf{portes d'une orbite}
            % \item propriété de \textbf{stabilité} et \textbf{transversalité} 
            % (respectivement \textbf{fortement}).
        \end{itemize}
    \end{block}
    \begin{figure}
        \includegraphics[width=\linewidth]{automate_porte.png}
    \end{figure}
\end{frame}

%------------------------------------------------

\begin{frame}
    \begin{block}{Propriétés d'automates}
        Certaine propriété des automates définie par M.Caron et M.Ziadi~:
        \begin{itemize}
            % \item les \textbf{portes d'une orbite}
            \item propriété de \textbf{stabilité} et \textbf{transversalité} 
            (respectivement \textbf{fortement}).
        \end{itemize}
    \end{block}
    \begin{figure}
        \includegraphics[width=\linewidth]{exemple_orbite.png}
    \end{figure}
\end{frame}

%------------------------------------------------

\begin{frame}
    \begin{block}{Glushkov et ces propriés}
        \begin{itemize}
            \item[\textbullet] Il existe un algorithme de transformation 
                d'expressions régulières en automate qui est celui de 
                \textit{Glushkov}
            
            \item[\textbullet] Toutes les orbites maximales des automates de 
                Glushkov sont fortement stables et transversal
        \end{itemize}
    \end{block}
\end{frame}

%------------------------------------------------

%------------------------------------------------

\begin{frame}
    \begin{block}{Automate}
        Automate de l'expression régulière \((a.b).a*.b*.(a+b)*\)~: 
    \end{block}
    \begin{figure}
        \includegraphics[width=\linewidth]{glushkov.png}
    \end{figure}
\end{frame}

%------------------------------------------------

% Plan : 
%   - présentation du langage
%   - expression, parseurs
%   - automate (parler de la représentation Json)
%   - (test)
%   - (benchmark mes deux implémentations)

%------------------------------------------------

\begin{frame}{Implémentation}
    \begin{textblock*}{20pt}(\textwidth-70pt,0pt)
        \includegraphics[width=100pt]{crabe.png}
    \end{textblock*}
    \begin{itemize}
        \item[\textbullet] Langage proche de la machine, au même niveau que le 
        \textit{C}
        \item[\textbullet] Crée en 2006 par un employer de chez Mozilla qui a 
            repris le projet après.
        \item[\textbullet] Avec des systèmes de sécurisation du code
        \item[\textbullet] Utilisé dans des domaines tels que~:
            \begin{itemize}
                \item[\textbullet] Développement de Systèmes
                \item[\textbullet] Développement d'application
                \item[\textbullet] Dans des appareils embarqués
            \end{itemize} 
    \end{itemize}
\end{frame}

%------------------------------------------------

\begin{frame}[fragile]
    \begin{textblock*}{20pt}(\textwidth-70pt,0pt)
        \includegraphics[width=100pt]{crabe.png}
    \end{textblock*}
    \begin{figure}  
        \begin{minted}{rust}
            #[derive(Clone, Debug, PartialEq)]
            pub enum RegExp<T> {
                Epsilon,
                Symbol(T),
                Repeat(Box<RegExp<T>>),
                Concat(Box<RegExp<T>>, Box<RegExp<T>>),
                Or(Box<RegExp<T>>, Box<RegExp<T>>),
            }
        \end{minted}
        \caption{
            Définition du type \mintinline{rust}{RegExp}, représentant une 
            expression rationnelle.
        }
    \end{figure}  
\end{frame}

%------------------------------------------------

%------------------------------------------------

\begin{frame}[fragile]
    \begin{textblock*}{20pt}(\textwidth-70pt,0pt)
        \includegraphics[width=100pt]{crabe.png}
    \end{textblock*}
    \begin{figure}  
        \begin{minted}{rust}
            #[derive(Debug)]
            pub struct State<'a, T, V>
            where
                T: Eq + Hash,
            {
                value: V,
                previous: HashMap<T, HashSet<RefState<'a, T, V>>>,
                follow: HashMap<T, HashSet<RefState<'a, T, V>>>,
            }
        \end{minted}
        \caption{
            Définition du type \mintinline{rust}{State}, représentant un 
            état.
        }
    \end{figure}  
\end{frame}

%------------------------------------------------


%------------------------------------------------

\begin{frame}[fragile]
    \begin{textblock*}{20pt}(\textwidth-70pt,0pt)
        \includegraphics[width=100pt]{crabe.png}
    \end{textblock*}
    \begin{figure}  
        \begin{minted}{rust}
            #[derive(Debug)]
            pub struct InnerAutomata<'a, T, V>
            where
                T: Eq + Hash + Clone,
            {
                states: HashSet<RefState<'a, T, V>>,
                inputs: HashSet<RefState<'a, T, V>>,
                outputs: HashSet<RefState<'a, T, V>>,
            }
        \end{minted}
        \caption{
            Définition du type \mintinline{rust}{InnerAutomata}, représentant
            un automate.
        }
    \end{figure}  
\end{frame}

%------------------------------------------------

%------------------------------------------------

\begin{frame}[fragile]
    \begin{textblock*}{20pt}(\textwidth-70pt,0pt)
        \includegraphics[width=100pt]{crabe.png}
    \end{textblock*}
    \begin{figure}  
        \begin{minted}{json}
            {
                "states":[6,5,3,4,2,1,0],
                "inputs":[0],
                "outputs":[6,2,3,4],
                "follows":[
                    [6,"a",5],
                    [5,"b",6],
                    [3,"b",4],
                    [3,"a",3],
                    [3,"a",5],
                    [4,"a",5],
                ]
            }
        \end{minted}
        \caption{
            Représentation d'un automate au format \textit{JSON} 
        }
    \end{figure}  
\end{frame}

%------------------------------------------------

\begin{frame}
    \begin{textblock*}{20pt}(\textwidth-70pt,0pt)
        \includegraphics[width=100pt]{crabe.png}
    \end{textblock*}
    \begin{figure}
        \includesvg[width=\linewidth]{rust_dot.svg}
        \caption{
            Représentation d'un automate au format \textit{Dot} 
        }
    \end{figure}
\end{frame}

%------------------------------------------------

%------------------------------------------------

\begin{frame}{Implémentation}
    \begin{textblock*}{20pt}(\textwidth-50pt,10pt)
        \includegraphics[width=70pt]{haskell.png}
    \end{textblock*}
    \begin{itemize}
        \item[\textbullet] Langage fonctionnel \textbf{pur}
        \item[\textbullet] Utilise des concepts de la 
        \textbf{théorie des catégories} (\textbf{Monade})
        \item[\textbullet] Très forte abstraction qui mène à une sûreté 
        d'exécution (Monade \mintinline{haskell}{IO})
        \item[\textbullet] Utilisé dans des domaines tels que~:
            \begin{itemize}
                \item[\textbullet] La finance
                \item[\textbullet] Le militaire
                \item[\textbullet] La recherche  
            \end{itemize} 
    \end{itemize}
\end{frame}

%------------------------------------------------

\begin{frame}[fragile]
    \begin{textblock*}{20pt}(\textwidth-50pt,10pt)
        \includegraphics[width=70pt]{haskell.png}
    \end{textblock*}
    \begin{figure}        
        \begin{minted}{haskell}
            data Exp a
                = Empty
                | Epsilon
                | Star (Exp a)
                | Plus (Exp a) (Exp a)
                | Point (Exp a) (Exp a)
                | Sym a
                deriving (Foldable, Functor, Traversable)
        \end{minted}
        \caption{
            Définition du type \mintinline{haskell}{Exp}, représentant une 
            expression rationnelle.
        }
    \end{figure}
\end{frame}

%------------------------------------------------

\begin{frame}[fragile]
    \begin{textblock*}{20pt}(\textwidth-50pt,10pt)
        \includegraphics[width=70pt]{haskell.png}
    \end{textblock*}
    \begin{figure}
        \begin{minted}{haskell}
            class NFA nfa where
                type StateType nfa :: Type
                type TransitionType nfa :: Type
                -- Exemple de fonction de cette classe de type 
                accept :: [TransitionType nfa] -> nfa -> Bool
                automatonToDot ::
                    (Show (StateType nfa), Show (TransitionType nfa))
                    => nfa
                    -> DotGraph Gr.Node
        \end{minted}
        \caption{Classe de type \mintinline{haskell}{NFA}.}
    \end{figure}
\end{frame}

%------------------------------------------------

\begin{frame}[fragile]
    \begin{textblock*}{20pt}(\textwidth-50pt,10pt)
        \includegraphics[width=70pt]{haskell.png}
    \end{textblock*}
    \begin{figure}
        \begin{minted}{haskell}
            data NFAF state transition = NFAF
                { sigma   :: Set.Set transition
                , etats   :: Set.Set state
                , premier :: Set.Set state
                , final   :: Set.Set state
                , delta :: state -> transition -> Set.Set state
                }
        \end{minted}
        \caption{Définition du type \mintinline{haskell}{NFAF}.}
    \end{figure}
\end{frame}

%------------------------------------------------

\begin{frame}[fragile]
    \begin{textblock*}{20pt}(\textwidth-50pt,10pt)
        \includegraphics[width=70pt]{haskell.png}
    \end{textblock*}
    \begin{figure}
        \begin{minted}{haskell}
            data NFAG state transition = NFAG
                { sigma   :: Set.Set transition
                , etats   :: Map.Map state Int
                , premier :: Set.Set Int
                , final   :: Set.Set Int
                , graph   :: Gr.Gr state transition
                , lastN   :: Int
                }
        \end{minted}
        \caption{Définition du type \mintinline{haskell}{NFAG}.}
    \end{figure}
\end{frame}

%------------------------------------------------

\begin{frame}[fragile]
    \begin{textblock*}{20pt}(\textwidth-50pt,10pt)
        \includegraphics[width=70pt]{haskell.png}
    \end{textblock*}
    \begin{figure}
        \begin{minted}{json}
                    {
                        "nodes": [0, 1, 2, 3],
                        "first": [0, 2],
                        "final": [1],
                        "transitions": [
                            [0, 1, "a"], 
                            [1, 3, "b"], 
                            [2, 0, "a"], 
                            [3, 2, "b"]] 
                    }
        \end{minted}
        \caption{Exemple de fichier \textit{JSON} représentant un automate.}
    \end{figure}
\end{frame}

%------------------------------------------------

\begin{frame}
    \begin{textblock*}{20pt}(\textwidth-50pt,10pt)
        \includegraphics[width=70pt]{haskell.png}
    \end{textblock*}
    \begin{figure}
        \includegraphics[width=\linewidth]{automate_haskell.png}
        \caption{
            Représentation graphique de l'automate du fichier \textit{JSON}.
        }
    \end{figure}
\end{frame}

%------------------------------------------------

\begin{frame}[fragile]
    \begin{textblock*}{20pt}(\textwidth-50pt,10pt)
        \includegraphics[width=70pt]{haskell.png}
    \end{textblock*}
    \begin{figure}
        \begin{minted}[frame=lines, fontsize=\tiny]{bash}
        $ stack test
        Testing NG.NFAG implementation...
        +++ OK, passed 100 tests.
        +++ OK, passed 100 tests.
        +++ OK, passed 100 tests.
        +++ OK, passed 100 tests.
        +++ OK, passed 100 tests.
        +++ OK, passed 100 tests.
        +++ OK, passed 100 tests.
        Testing NF.NFAF implementation...
        +++ OK, passed 100 tests.
        +++ OK, passed 100 tests.
        +++ OK, passed 100 tests.
        +++ OK, passed 100 tests.
        +++ OK, passed 100 tests.
        +++ OK, passed 100 tests.
        +++ OK, passed 100 tests.
        
        Testing Glushkov properties...
        +++ OK, passed 100 tests.
        
      \end{minted}
        \caption{
            Résultat des tests par propriétés des types 
            \mintinline{haskell}{NFAG} et \mintinline{haskell}{NFAF}.
        }
    \end{figure}
\end{frame}

%------------------------------------------------

\begin{frame}[fragile]
    \begin{textblock*}{20pt}(\textwidth-50pt,10pt)
        \includegraphics[width=70pt]{haskell.png}
    \end{textblock*}
    \begin{center}
        \href{run:report_perf.html}{Rapport du Benchmark}
    \end{center}
\end{frame}

%------------------------------------------------

\begin{frame}
    \begin{textblock*}{20pt}(\textwidth-130pt,10pt)
        \includegraphics[width=160pt]{haskell_rust.png}
    \end{textblock*}
    \Huge{\centerline{\textbf{Démonstration des applications.}}}
\end{frame}

%------------------------------------------------

\begin{frame}
    \Huge{\centerline{\textbf{Bilan}}}
\end{frame}

%------------------------------------------------

\begin{frame}
    \Huge{\centerline{\textbf{Merci de nous avoir écoutés.}}}
\end{frame}

%------------------------------------------------

\end{document}