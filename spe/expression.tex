\section{Expressions régulières}

\begin{Definition}
  Un \textbf{alphabet} est un ensemble fini de symbole appelé \textbf{lettre}.
  On prend souvent la notation \(\Sigma\) pour représenter un \textbf{alphabet}.
\end{Definition}

\begin{Definition}
  Un \textbf{langage} sur un alphabet \(\Sigma\), noté \(L\) est un 
  sous-ensemble de \(\Sigma^*\).
\end{Definition}

\begin{Definition}
  Soit \(\Sigma\) un alphabet, les \textbf{expressions régulières} sur 
  \(\Sigma\) sont définis récursivement comme suit.

  \noindent\textbf{Cas de base~:}

  \begin{itemize}
    \item \(\emptyset\), est l'\textbf{expression} représentant l'ensemble vide,
    \item \(\epsilon\) est l'\textbf{expression} représentant l'ensemble 
    \(\{\epsilon\}\),
    \item \(\forall a \in \Sigma\), \(a\) est l'\textbf{expression} représentant 
    l'ensemble \(\{a\}\),
  \end{itemize}

  \noindent\textbf{Cas héréditaire~:}

  Soit \(r, r'\) deux \textbf{expressions régulières} représentant 
  respectivement les langages \(R, R' \in (\Sigma^*)^2\), on a définie les 
  opérateurs ci-dessous sur l'ensemble des \textbf{expressions régulières} de 
  \(\Sigma\) que l'on note \(E_{\Sigma}\), 

  \begin{itemize}
    \item \(r + r'\) représente le langage dénoté par \(R \cup R'\),
    \item \(r . r'\) représente le langage dénoté par \(R . R'\),
    \item \(r^*\) représente le langage dénoté par \(R^*\).
  \end{itemize}
\end{Definition}

\begin{Definition}
  Sera noté \(E_{(\Sigma, \mathbb{N})}\), l'ensemble des 
  \textbf{expressions régulières} sur l'alphabet \(\Sigma\) au quelle aura été 
  associé à chaque lettre son indice d'apparition dans l'ordre de lecture 
  gauche-droite de chacune des expressions.
\end{Definition}

\begin{Definition}
  Avec \(r \in E_{Q}\), on note \(L(r)\), le \textbf{langage} représenté par 
  l'\textbf{expression régulière} \(r\).
\end{Definition}

\begin{Definition}
  On définit la fonction \textbf{linéarisation} de la façon suivante,
  \[
    linearisation :: E_{\Sigma} \to E_{(\Sigma, \mathbb{N})}
  \]
\end{Definition}

\begin{Definition}
  Sur une \textit{expression régulière} \(r \in E_{Q}\), on note la fonction 
  renvoyant l'ensemble des premières lettres du langage représenter par \(r\) 
  comme étant \(first(r)\). Cette fonction est définie de récursivement comme 
  suit~:

  \begin{equation*}
    \begin{split}
    & first(\emptyset) = first(\epsilon) = \emptyset\\
    & first(x) = \{x\}, x \in Q\\
    & first(f + g) = first(f) \cup first(g)\\
    & first(f.g) = \left \{
      \begin{array}{r c l}
        first(f), \text{ si } \epsilon \notin L(f)\\
        first(f) \cup first(g), \text{ si } \epsilon \in L(f)\\
      \end{array}
      \right . \\
    & first(f^*) = first(f)
    \end{split}
  \end{equation*}
  Avec \((f,g) \in (E_{Q})^2\)
\end{Definition}

\begin{Definition}
  De même, on définit la fonction \(last(r)\) permettant de renvoyer à partir 
  d'une \textbf{expression régulière} l'ensemble des dernières lettres de tous 
  les mots du langage représenter par \(r\).
  Tout comme la fonction \(first\), \(last\) est définie de la même façon 
  récursivement sauf pour le cas suivant~:

  \[
    last(f.g) = \left \{
      \begin{array}{r c l}
        last(g), \text{ si } \epsilon \notin L(g)\\
        last(f) \cup last(g), \text{ si } \epsilon \in L(g)\\
      \end{array}
      \right . \\
  \]
\end{Definition}

\begin{Definition}
  Nous définissons là fonction \(index\), comme prenant en paramètre une 
  expression régulière indexé et retournant une table permettant d'obtenir la 
  fonction suivante~:
  \[
    lookup :: Map(Int, \Sigma) \to Int \to Maybe(\Sigma) 
  \]
  Renvoyant \(Nothing\), en cas de nom présence de l'indice passer en paramètre 
  dans l'expression, sinon renvoie \(Just\) la valeur d'indice associé dans 
  l'expression.
\end{Definition}

\begin{Definition}
  On définit la fonction \(follow\), comme un moyen d'obtenir les possibles 
  symboles qui peuvent suivre une certaine lettre dans une 
  \textbf{expression régulière}. Cette fonction à la signature suivante~:
  \[
    follow :: E_{(\Sigma, \mathbb{N})} \to (Int \to 2^{\Sigma \times \mathbb{N}})
  \]
  Retournant pour l'indice d'un élément d'une \textbf{expression régulière}, 
  l'ensemble des éléments indicé qui peuvent le suivre.
\end{Definition}