\documentclass{article}
\usepackage{amsfonts}
\usepackage{amsmath}

\begin{document}

Soit \(\Sigma, Q\) des ensembles finies. On notera, \(E_{\Sigma}\) l'ensemble 
des d'expression régulière sur \(\Sigma\)

Sera noté \(E_{(Q, \mathbb{N})}\), l'ensemble des expressions régulières de 
\(E_Q\) auquelle on aura associé à chaque charactère son indice d'apparition 
dans l'ordre de l'écture gauche-droite en commensant à un de l'expression.

Soit \(R\) un ensemble finie et \(e \in E_R\),

\begin{equation*}
  \begin{split}
  & linearisation(e) :: E_R \to E_{(R, \mathbb{N})} \\
  & alphabet(e) :: E_R \to V \text{, avec } V \subset Q
  \end{split}
\end{equation*}

La fonction \(alphabet\) renvoie le sous ensemble de \(R\) correspondant à tous 
les symboles apparaissant au moins une fois dans \(e\).

\begin{equation*}
  \begin{split}
  & first(e) :: E_Q \to F \text{, tel que } F \subset Q \\
  & last(e) :: E_Q \to F \text{, tel que } F \subset Q
  \end{split}
\end{equation*}

La fonction qui permet de récuprer la lettre associé a son indice pour une 
expression indicé est définie de la façon suivante~:

\[
indexE :: E_{(Q, \mathbb{N})} \to \mathbb{N} \to Q
\]

\end{document}