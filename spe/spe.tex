\documentclass{article}
\usepackage{amsfonts}
\usepackage{amsmath}

\begin{document}

\section{Expression}

Soit \(\Sigma, Q\) des ensembles finies. On notera, \(E_{\Sigma}\) l'ensemble 
des d'expression régulière sur \(\Sigma\)

Sera noté \(E_{(Q, \mathbb{N})}\), l'ensemble des expressions régulières de 
\(E_Q\) auquelle on aura associé à chaque charactère son indice d'apparition 
dans l'ordre de l'écture gauche-droite en commensant à un de l'expression.

Soit \(R\) un ensemble finie et \(e \in E_R\),

\begin{equation*}
  \begin{split}
  & linearisation(e) :: E_R \to E_{(R, \mathbb{N})} \\
  & alphabet(e) :: E_R \to V \text{, avec } V \subset Q
  \end{split}
\end{equation*}

La fonction \(alphabet\) renvoie le sous ensemble de \(R\) correspondant à tous 
les symboles apparaissant au moins une fois dans \(e\).

\begin{equation*}
  \begin{split}
  & first(e) :: E_Q \to F \text{, tel que } F \subset Q \\
  & last(e) :: E_Q \to F \text{, tel que } F \subset Q \\
  & follow(e) :: E_{(Q, \mathbb{N})} \to \mathbb{N} \to S 
  \end{split}
\end{equation*}
avec \(S\) représentant l'ensemble des symboles qui peuvent suivre le symbole 
d'indice donner.

La fonction qui permet de récuprer la lettre associé a son indice pour une 
expression indicé est définie de la façon suivante~:

\[
indexE :: E_{(Q, \mathbb{N})} \to \mathbb{N} \to Q
\]

Nous représenterons un automate de la façon suivante~: 
\(M = (\Sigma, Q, P, F, \delta)\) avec \(\Sigma\), l'ensemble d'éléments pouvant 
être labelle d'une transition~; 
\(Q\) l'ensemble des états de l'automate~; 
\(P\) l'ensemble des états initiaux de l'automate tel que \(P \subset Q\)~; 
\(F\) l'ensemble des états finaux de l'automate tel que \(F \subset Q\)~; 
\(delta\), une fonction tel que
\[
  delta :: Q \to \Sigma \to Q
\]

De cette définition viens la fonction~:
\[
  glushkov :: E_{\Sigma} \to M \text{ avec, } M = (\Sigma, Q, P, F, \delta)
\]
qui permet la transformation d'une expression rationnelle en automate de 
\textit{glushkov}.

% expliquer la fonction Follow qui n'est pas réelement la fonction Follow.

\section{Automate}

À un automate quelquonc \(M\), on peut appliquer les opérations d'ajout (resp. 
supression) de tansition et d'état. On définit un orbite noté \(\mathcal{O}\), 
comme étant un sous-ensemble de \(Q\) tel que pour \((x, x') \in Q^2\), il 
existe une suite de transition partant de \(x\) vers \(x'\).

Un orbit est dit \textit{maximal} si et seulement si, pour tout 
\(x \in \mathcal{O}\) et \(x' \notin \mathcal{O}\), il n'existe qu'une suite de 
transition de \(x\) vers \(x'\) ou de \(x'\) vers \(x\).

On définit l'automate d'un orbite noté \(M_{\mathcal{O}}\), tel que 
\(M_{\mathcal{O}} = (\Sigma, Q \cap \mathcal{O}, P \cap \mathcal{O}, F \cap \mathcal{O}, 
\delta')\) avec \(s \in Q, t \in \Sigma\)~:

\[
  \delta'(s, t) =    \left \{
    \begin{array}{r c l}
      delta(s, t) \cap \mathcal{O} \text{, si } s \in Q \\ 
      \emptyset \text{, sinon} \\
    \end{array}
    \right .
\]

Soit \(x \in Q\), on note \(Q^{+}(x)\) (respectivement \(Q^{-}(x)\)) les 
succéseurs (resp. prédécésseur) directe de l'état \(x\). 


\end{document}