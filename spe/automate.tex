\section{Automate fini}

\begin{Definition}
  On définit un \textbf{automate fini}, comme étant un cinq-uplet. On note 
  généralement l'\textbf{automate} \(M\), \(M = (\Sigma, Q, I, F, \delta)\).
  \begin{itemize}
    \item \(\Sigma\), est l'alphabet d'entrée,
    \item \(Q\), l'ensemble des états,
    \item \(I\), est un sous-ensemble de \(Q\), il s'agit des états initiaux de 
    l'\textbf{automate},
    \item \(F\), est un sous-ensemble de \(Q\), il s'agit des états finaux de 
    l'\textbf{automate},
    \item \(\delta\) est la fonction de transition définie de la façon 
    suivante~:
    \[
      \delta :: Q \times \Sigma \to 2^Q
    \]
  \end{itemize}
\end{Definition}

\begin{Definition}
  Un mot est accepté par un \textbf{automate}, \(M = (\Sigma, Q, I, F, \delta)\) 
  si et seulement si, il existe au moins une suite de d'état tel que 
  \(m = (p_0, \dots, p_{n-1})\) tel que \(p_0 \in I\) et \(p_{n-1} \in F\).
\end{Definition}

\begin{Definition}
  On définit le langage d'un automate \(M = (\Sigma, Q, I, F, \delta)\), comme 
  étant \(L(M) = \{w \in \Sigma^* | w \text{ est reconnu par } M\}\)
\end{Definition}

\begin{Definition}
  Un automate est dit \textbf{homogène} si et seulement si toutes les 
  transitions qui arrivent sur un état sont sur le même symbole de l'alphabet.
\end{Definition}

\begin{Definition}
  Un automate \(M = (\Sigma, Q, I, F, \delta)\) est dit \textbf{standard} si et 
  seulement si~:
  \begin{itemize}
    \item \(|I| = 1\)
    \item \(\{q \in Q | \exists a \in \Sigma, \delta(q, a) \in I\} = \emptyset\)
  \end{itemize}
\end{Definition}

\begin{Definition}
  On peut supprimer d'un automate \(M = (\Sigma, Q, I, F, \delta)\) un 
  \textbf{état}. À l'aide de la fonction~:
  \[
    removeState((\Sigma, Q, I, F, \delta), q) = (\Sigma, Q/\{q\}, I/\{q\}, 
    F/\{q\}, \delta') 
  \]
  Avec \(\delta'(e, a) = \left \{
    \begin{array}{r c l}
      \emptyset, \text{ si } e = q\\
      \delta(e, a)/\{q\}, \text{ sinon } \\
    \end{array}
    \right . \)
  On peut aussi supprimer une \textbf{transition} \(q, q', a\) avec \((q, q') 
  \in Q^2, a \in \Sigma\). 
  \[
    removeTransition((\Sigma, Q, I, F, \delta), (q, q', a)) = 
    (\Sigma, Q, I, F, \delta')
  \]
  Avec \(\delta'(e, a') = \left \{
    \begin{array}{r c l}
      \delta(e, a)/\{q'\}, \text{ si } e = q \land a' = a\\
      \delta(e, a), \text{ sinon } \\
    \end{array}
    \right .\)
  On éteint cette suppression à la surpression de toutes les transitions entre 
  deux n\oe uds de la façon suivante~:
  \[
    removeTransitions((\Sigma, Q, I, F, \delta), (q, q')) =
    (\Sigma, Q, I, F, \delta')
  \]
  Avec \(\delta'(e, a') = \left \{
    \begin{array}{r c l}
      \delta(e, a)/\{q'\}, \text{ si } e = q\\
      \delta(e, a), \text{ sinon } \\
    \end{array}
    \right .\)
\end{Definition}

\begin{Remark}
  Les fonctions \(removeState, removeTransition\) et \(removeTransitions\)
  augment le nombre d'opérations faites sur la fonction \(\delta\) de 
  l'automate. Il serait donc intéressant dans le cas d'une implémentation de 
  cette méthode, de donner à cette implémentation une opération de 
  \textbf{mémoization} qui permettrait quand voulu de réduire le temps de calcul 
  de la fonction \(\delta\).
\end{Remark}

\begin{Definition}
  On définit la fonction \(\Omega^{+}(q)\) (respectivement \(\Omega^{-}(q)\)) 
  sur l'ensemble des états d'un automate \(M = (\Sigma, Q, I, F, \delta)\), 
  comme étant l'ensemble des directes successeurs (respectivement prédécesseur) 
  d'un état \(q\) tel que \(q \in Q\).
\end{Definition}

\begin{Definition}
  On peut construire en extrayant d'un automate tous les états n'appartenant pas 
  à une liste d'état. On appellera cette automate, un 
  \textbf{sous-automate d'état}. On définit alors la fonction 
  \(extractListStateAutomata\), permettant d'obtenir le \(sous-automate\) de 
  \(M = (\Sigma, Q, I, F, \delta)\) ne contenant que les états de la liste 
  \(L = (q_0, \dots, q_{n-1})\).
  \[
    extractListStateAutomata(M, L) = (\Sigma, Q \cap L, I \cap L, F \cap L, 
    \delta')
  \]
  Avec \(\delta'(q, a) = \left \{
    \begin{array}{r c l}
      \emptyset, \text{ si } q \notin L\\
      \delta(q, a) \cup L, \text{ sinon } \\
    \end{array}
    \right .\)
\end{Definition}

\begin{Definition}
  Soit un automate \(M = (\Sigma, Q, I, F, \delta)\), une \textbf{orbite} est un 
  sous-ensemble de \(Q\) tel que, 
  \[
    \forall (q, q') \in Q^2, \text{ il existe un chemin non trivial de } q 
    \text{ vers } q' \text{ et de } q' \text{ vers } q.
  \]
\end{Definition}

\begin{Definition}
  Une \textbf{orbite} d'un automate \(M = (\Sigma, Q, I, F, \delta), \mathcal{O} 
  \subset Q\) est dite \textbf{maximal} si et seulement si, 
  \(\forall q \in \mathcal{O}, \forall q' \notin \mathcal{O}\), il n'existe pas 
  en même temps de chemin de q vers q' et de q' vers q.  
\end{Definition}

\begin{Definition}
  Pour une \textbf{orbite} \(\mathcal{O}\) d'un automate \(M = (\Sigma, Q, I, F, 
  \delta)\), on définit les \textbf{portes d'entrée} (respectivement de 
  \textbf{sortie}), noté \(In(\mathcal{O})\) (respectivement 
  \(Out(\mathcal{O})\)) de la façon suivante~:
  \[
    In(\mathcal{O}) = \{x \in \mathcal{O} | x \in I \lor \Omega^{-}(x) \neq
    \emptyset\}\\
  \]
  \[
    Out(\mathcal{O}) = \{x \in \mathcal{O} | x \in F \lor \Omega^{+}(x) \neq
    \emptyset\}
  \]
\end{Definition}

\begin{Definition}
  Une \textbf{orbite} \(\mathcal{O}\) d'un automate \(M = (\Sigma, Q, I, F, 
  \delta)\) est dit \textbf{stable} si et seulement si, \(In(\mathcal{O}) 
  \times Out(\mathcal{O}) \subset \{(q, q') | \forall q \in Q, \forall a \in 
  \Sigma, \delta(q, a) = q'\}\).
\end{Definition}

\begin{Definition}
  Une \textbf{orbite maximale} \(\mathcal{O}\) d'un automate \(M = (\Sigma, Q, 
  I, F, \delta)\) est dite \textbf{hautement stable} si et seulement si, 
  \(\mathcal{O}\) est stable et qu’après avoir supprimé toutes les transitions 
  des états présents dans \(In(\mathcal{O}) \times Out(\mathcal{O})\), chaque 
  sous-orbite maximal obtenu est \textbf{hautement stable}.
\end{Definition}

\begin{Definition}
  Une \textbf{orbite} \(\mathcal{O}\) d'un automate \(M = (\Sigma, Q, I, F, 
  \delta)\) est dite \textbf{transverse} si et seulement si, \(\forall (q, q') 
  \in Out(\mathcal{O}), \Omega^{+}(x) = \Omega^{+}(y)\) et \(\forall (q, q') \in 
  In(\mathcal{O}), \Omega^{-}(x) = \Omega^{-}(y)\).
\end{Definition}

\begin{Definition}
  Une \textbf{orbite maximale} \(\mathcal{O}\) d'un automate \(M = (\Sigma, Q, 
  I, F, \delta)\) est dite \textbf{hautement transverse} si \(\mathcal{O}\) est 
  \textbf{transverse} et que chaque sous-orbite \textbf{maximale} est 
  \textbf{hautement transverse} une fois les transitions d'états 
  \(In(\mathcal{O}) \times Out(\mathcal{O})\) supprimer.
\end{Definition}